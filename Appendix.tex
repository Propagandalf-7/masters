\documentclass[main.tex]{subfiles}
\begin{document}
\chapter*{Appendix}
\addcontentsline{toc}{chapter}{Appendix}

\section*{List of symbols}
\addcontentsline{toc}{section}{List of symbols}

This list of symbols contain some of the most used symbols in this dissertation. The symbols are grouped into three categories: Vector Spaces and Related Concepts, Mathematical Measures and Operations, and Physical Quantities and Parameters. The first appearance of each symbol is also listed.

Other symbols that are section specific or that are contextually evident might not be in this list.

% Vector Spaces and Related Concepts
\noindent
\textbf{Vector Spaces and Related Co
ncepts}\\
\begin{tabularx}{\textwidth}{cXr}
\toprule
Symbol & Description & First Appearance (Page) \\ 
\midrule
\( \mathbb{N} \) & set of natural numbers \( j \) & \pageref{sym:natural} \\
\( R^n \) & n-dimensional space of real numbers & \pageref{sym:Rn} \\
\( \Omega \) & subset of \( R^n \), usually representing a body/reference configuration & \pageref{sym:Omega} \\
\( \partial \Omega \) & the boundary of $\Omega$ & \pageref{sym:boundaryOmega} \\
\( \bar{\Omega} \) & the boundary of $\Omega$ & \pageref{sym:baromega} \\
\( C^n \) & set of n-times continuously differentiable functions & \pageref{sym:Cn} \\
\( C^\infty \) & space of smooth functions & \pageref{sym:Cinfty} \\
\( C^\infty_0 \) & space of smooth functions with compact support & \pageref{sym:Cinftyzero} \\
\( L^n \) & space of n-power Lebesgue integrable functions & \pageref{sym:Ln} \\
\( H^n \) & space of functions with weak derivatives up to order n (n'th dimensional Sobolev space) & \pageref{sym:Hn} \\
\( X \) & global space & \pageref{sym:Xspace} \\
\( V \) & inertia space & \pageref{sym:Wspace} \\
\( W \) & energy space & \pageref{sym:Vspace} \\
\( T(\Omega) \) & test function space on \( \Omega \) & \pageref{sym:TOmega} \\
\( S^h \) & finite dimensional subspace & \pageref{sym:Sh1} \\
\( \mathcal{P}_j \) & set of all polynomials of degree at most \( j \) & \pageref{sym:poly} \\
\( E_n \) & space spanned by the orthonormal basis vectors $e_i$ & \pageref{sym:En} \\


\bottomrule
\end{tabularx}

% Mathematical Measures and Operations
\noindent
\textbf{Mathematical Measures and Operations}\\
\begin{tabularx}{\textwidth}{cXr}
\toprule
Symbol & Description & First Appearance (Page) \\ 
\midrule
\( a(\cdot,\cdot), b(\cdot,\cdot), c(\cdot,\cdot) \) & bilinear forms & \pageref{sym:a}, \pageref{sym:b}, \pageref{sym:c} \\
\( (\cdot, \cdot)_X \) & innerproduct of X & \pageref{sym:innerproduct} \\
\( ||\cdot||_X \) & norm in the space \( X \) & \pageref{sym:norm} \\
\( \partial_x^nf \) & n-th partial derivative of f with respect to \( x \) & \pageref{sym:partial_diff} \\
\( \textrm{div} X \) & divergence of the matrix \( X \) & \pageref{sym:divT} \\
\( \textrm{Tr}(X) \) & Trace of the matrix \( X \) & \pageref{sym:TrT} \\
\( \textrm{det}(M) \) & determinant of M & \pageref{sym:determinant} \\
\( \textrm{span}(\cdot) \) & span of a set & \pageref{sym:span} \\
\( dV \) & volume integral measure & \pageref{sym:dV} \\
\( dS \) & surface integral measure & \pageref{sym:dS} \\
\( dA \) & area integral measure & \pageref{sym:dA} \\
\( ds \) & line integral measure & \pageref{sym:ds} \\
\( \mathcal{E}(\cdot) \) & energy function & \pageref{sym:Energy} \\
\( R(\cdot) \) & Rayleigh quotient & \pageref{sym:Rayleigh} \\
\( \Pi \) & interpolation operator & \pageref{sym:interpolation} \\
\( \bar{u} \) & another form explicity showing u is a vector & \pageref{sym:baru} \\
\( \textrm{RE} \) & abbriviation for the relative error & \pageref{sym:RE} \\
\bottomrule
\end{tabularx}

% Physical Quantities and Parameters
\noindent
\textbf{Physical Quantities and Parameters}\\
\begin{tabularx}{\textwidth}{cXr}
\toprule
Symbol & Description & First Appearance (Page) \\ 
\midrule
\( \lambda \) & eigenvalue & \pageref{sym:lambda} \\
\( u \) or \( w \) & displacement vector & \pageref{sym:u} \\
\( \phi \) & arbitrary vector/rotation of cross-section of Timoshenko beam & \pageref{sym:phi} \\
\( Q \) & force per unit volume & \pageref{sym:Q} \\
\( \rho \) & density & \pageref{sym:rho} \\
\( T \) & stress tensor & \pageref{sym:T} \\
\( \sigma_{ij} \) & element of the stress tensor \( T \) & \pageref{sym:sigmaij} \\
\( \mathcal{E} \) & infinitesimal strain tensor & \pageref{sym:mathcalE} \\
\( \varepsilon_{ij} \) & element of the infinitesimal strain tensor \( \mathcal{E} \) & \pageref{sym:varepsilonij} \\
\( E \) & Young's modulus & \pageref{sym:E} \\
\( \nu \) & Poisson's ratio & \pageref{sym:nu} \\
\( t \) & time & \pageref{sym:t} \\
\( \ell \) & dimension representing length & \pageref{sym:ell} \\
\( h \) & dimension representing height & \pageref{sym:height} \\
\( b \) &  dimension representing width & \pageref{sym:width} \\
\( G \) & shear modulus of elasticity & \pageref{sym:G} \\
\( A \) & area of a cross-section & \pageref{sym:A} \\
\( V \) & shear force & \pageref{sym:V} \\
\( I \) & area moment of inertia & \pageref{sym:Iinertia} \\
\( M \) & moment & \pageref{sym:M} \\
\( f^* \) & dimensionless form of f & \pageref{sym:sigmaijstar} \\
\( \tau \) & dimensionless time & \pageref{sym:tau} \\
\( \xi \) & dimensionless space & \pageref{sym:xi} \\
\( \alpha/\beta \) & dimensionless constants & \pageref{sym:alphabeta} \\
\( \kappa^2 \) & some dimensionless constant/shear correction factor & \pageref{sym:kappa2} \\
\( I \) & identity matrix & \pageref{sym:I} \\
\( \gamma \) & a dimensionless constant & \pageref{sym:gamma} \\
\( n \) & a normal vector & \pageref{sym:n} \\
\( \Sigma/\Gamma \) & distinct parts of \( \Omega \) & \pageref{sym:SigmaGamma} \\
\( \mu \) & eigenvector & \pageref{sym:mu} \\
\( e_i \) & orthonormal basis vector & \pageref{sym:e_i} \\

\bottomrule
\end{tabularx}

\section*{Sobolev spaces}
\addcontentsline{toc}{section}{Sobolev spaces}

\subsection*{The Space $L^2$}
Consider a measurable space $X$. The set of square integrable functions is called the $L^2$ space.\\

The inner product of $L^2$ is defined as 
\begin{eqnarray*}
	(f,g) = \int_X fg \ \ \ \ \textrm { for } f,g \in L2.
\end{eqnarray*}
The norm can be defined as $||f|| = (f,f)^{\frac{1}{2}}$ for each $f \in L^2(X)$. For reference, see \cite{Rud53}.

\subsection*{The Space \( L^p \)}
Consider a measurable space \( X \). For a real number \( p \geq 1 \), the set of \( p \)-integrable functions is called the \( L^p \) space. A function \( f \) belongs to \( L^p(X) \) if the \( p \)-th power of its absolute value is Lebesgue integrable, that is, if
\begin{eqnarray*}
    \int_X |f|^p < \infty.
\end{eqnarray*}

The \( L^p \) norm (or \( p \)-norm) is defined as
\begin{eqnarray*}
    ||f||_p = \left( \int_X |f|^p \right)^{\frac{1}{p}}
\end{eqnarray*}
for each \( f \in L^p(X) \).

\subsection*{Continuous function spaces}
$C^m(a,b)$ is the space of functions with continuous derivatives up to order $m$ over the open interval (a,b).

$C^m[a,b]$ is the space of functions in $C^m(a,b)$, with existing right derivatives at $a$ and existing left derivatives at $b$, up to order m.

$C_0^m(a,b)$ contains all functions $f$ in $C^m[a,b]$ with the property that there exists numbers $a < \alpha < \beta < b$ such that $f$ is zero on $[a,\alpha] \cup [\beta, b]$. This property is called compact support.

$C^\infty(a,b)$ contains all functions in $C^m(a,b)$ for all $m$.

$C^\infty[a,b]$ contains all functions in $C^m[a,b]$ for all $m$.

$C_0^\infty(a,b)$ contains all functions in $C_0^m(a,b)$ for all $m$.


\subsection*{First order weak derivative}
Suppose $u \in L^2(a,b)$ and there exist a $v \in L^2(a,b)$ such that
\begin{eqnarray*}
	(u,\phi') = -(v,\phi) \ \textrm{ for each } \phi \in C^{\infty}_0(a,b)
\end{eqnarray*}
then $v$ is called the first order weak derivative of $u$ and is denoted by $Du$.

\subsection*{Higher order weak derivative}
Suppose $u \in L^2(a,b)$ and there exist a $v \in L^2(a,b)$ such that
\begin{eqnarray*}
	(u,\phi^{(m)}) = (-1)^{(m)}(v,\phi) \ \textrm{ for each } \phi \in C^{\infty}_0(a,b)
\end{eqnarray*}
then $v$ is called the m'th order weak derivative of $u$ and is denoted by $D^{(m)}u$.

\subsection*{Sobolev spaces}

$W^n$ is the space of functions with weak derivatives up to order $n$. There are also special notation $W^{n,p}$ that indicates that the functions are P-intergrable. 

$H^n$ is the space of functions with weak derivatives up to order $n$ and the functions are square integrable. (i.e. $H^n = W^{n,2}$)

\chapter*{MATLAB Code}
\addcontentsline{toc}{section}{MATLAB Code}
\section*{Example code for Timoshenko beam model}
\lstinputlisting{code/timoshenko_beam/TimoshenkoEig.m}

\section*{Example code for two-dimensional elastic body using bi-cubics}
\lstinputlisting{code/two_dimensional_model/TwoDimensionalCantileverCubic.m}

\section*{Example code for Reissner-Mindlin plate model using bi-cubics}
\lstinputlisting{code/plate_model/PlateCantileverCubic.m}

\section*{Example code for three-dimensional elastic body using tri-cubics}
\lstinputlisting{code/three_dimensional_model/Copy_of_CubePC.m}


\end{document}
