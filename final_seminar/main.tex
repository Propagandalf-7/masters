\documentclass[8pt]{beamer}
\usetheme{metropolis} % Use the Metropolis theme
\usepackage{pgfplots}



\title{Comparison of linear beam models}
\author{Rudi du Toit}
\date{September 1, 2023}
\institute{University of Pretoria}

\begin{document}

% Redefine the title page template
\setbeamertemplate{title page}{
    \begin{center}
        \usebeamercolor[fg]{titlegraphic}\inserttitlegraphic\par
        \vskip0.25em%
        {\fontsize{14}{16}\selectfont\usebeamerfont{title}\usebeamercolor[fg]{title}\inserttitle\par}% Increase the font size of the title
        \ifx\insertsubtitle\@empty%
        \else%
        \vskip0.25em%
        {\fontsize{12}{14}\selectfont\usebeamerfont{subtitle}\usebeamercolor[fg]{subtitle}\insertsubtitle\par}% Increase the font size of the subtitle
        \fi%     
        \vskip1em\par
        \usebeamerfont{author}\insertauthor\par
        \vskip1em\par
        \usebeamerfont{institute}\insertinstitute\par
        \vskip1em%
        \includegraphics[width=0.2\linewidth]{logo-up.jpg}\par % Include the logo here
        \vskip1em\par
        \usebeamerfont{date}\insertdate\par
        \vfill
    \end{center}
}

\begin{frame}[plain]
\titlepage
\end{frame}

\begin{frame}{Introduction}
    \begin{itemize}
        \item Beams are widely used in engineering.
        \item A three-dimensional beam model would be most realistic representation of a real world beam.
        \item Three-dimensional beams are complex and require a lot of compute power, especially when using multiple beams in a single structure.
        \item Therefore simplified beam models are used, like the Timoshenko beam model.
        \item When in application is a Timoshenko beam a suitable replacement for a three-dimensional beam?
    \end{itemize}
\end{frame}

\begin{frame}{Introduction}
    \begin{itemize}
        \item Vibration problems are common in beam models.
        \item The validity of a Timoshenko beam model in application is an important consideration.
        \item Validity refers to how well the solution of a model compares to a more realistic, higher dimensional model.
    \end{itemize}
\end{frame}



\end{document}
