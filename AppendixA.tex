\documentclass[../../main.tex]{subfiles}
\begin{document}
\subsection*{The Space $L^2$}
Consider a measurable space $X$. The set of square integrable functions is called the $L^2$ space.\\

The inner product of $L^2$ is defined as 
\begin{eqnarray*}
	(f,g) = \int_X fg \ \ \ \ \textrm { for } f,g \in L2.
\end{eqnarray*}
The norm can be defined as $||f|| = (f,f)^{\frac{1}{2}}$ for each $f \in L^2(X)$. For reference, see \cite{Rud53}.\\


\textcolor{red}{***********************************************}\\

The space $C^m(a,b)$ is the space of functions with continuous derivatives up to order $m$ over the open interval (a,b).\\

The space $C^m[a,b]$ is the space of functions in $C^m(a,b)$, with existing right derivatives at $a$ and existing left derivatives at $b$, up to order m.\\

The space $C_0^m(a,b)$ contains all functions $f$ in $C^m[a,b]$ with the property that there exists numbers $a < \alpha < \beta < b$ such that $f$ is zero on $[a,\alpha] \cup [\beta, b]$.\\

The space $C^\infty(a,b)$ contains all functions in $C^m(a,b)$ for all $m$.\\

The space $C^\infty[a,b]$ contains all functions in $C^m[a,b]$ for all $m$.\\

The space $C_0^\infty(a,b)$ contains all functions in $C_0^m(a,b)$ for all $m$.\\

\textbf{Definition:} First Order Weak Derivative\\
Suppose $u \in L^2(a,b)$ and there exist a $v \in L^2(a,b)$ such that
\begin{eqnarray*}
	(u,\phi') = -(v,\phi) \ \textrm{ for each } \phi \in C^{\infty}_0(a,b)
\end{eqnarray*}
then $v$ is called the first order weak derivative of $u$ and is denoted by $Du$.\\

\textbf{Definition:} Higher Order Weak Derivative\\
Suppose $u \in L^2(a,b)$ and there exist a $v \in L^2(a,b)$ such that
\begin{eqnarray*}
	(u,\phi^{(m)}) = (-1)^{(m)}(v,\phi) \ \textrm{ for each } \phi \in C^{\infty}_0(a,b)
\end{eqnarray*}
then $v$ is called the m'th order weak derivative of $u$ and is denoted by $D^{(m)}u$.\\

Define inner product $$(u,v)_m = (u,v) + (Du,Dv) + ... + (D^mu ,D^mv)$$.

The space $W^1(a,b) \subset L^2(a,b)$ contains the functions with first order weak derivatives. $W^m(a,b)\subset L^2(a,b)$ contains the functions with weak derivatives up to order $m$. $W^m(a,b)$ is a Sobolev space with inner product $(u,v)_m$ and norm defined by 
\begin{eqnarray*}
	||u||_m = (u,u)^{\frac{1}{2}}_m.
\end{eqnarray*}

The space $H^1(a,b)$ is the closure of $C^1[a,b]$ in $W^1(a,b)$ with respect to the norm of $W^1(a,b)$. The space $H^m(a,b)$ is the closure of $C^m[a,b]$ in $W^m(a,b)$ with respect to the norm of $W^m(a,b)$.

\section*{Results for One-Dimensional Case}



\section*{Results for Higher-Dimensional Case}

\newtheorem{thmAH1}{Theorem}
\begin{thmAH1}
	The m'th order weak derivative $D^{(m)}u$ is uniquely determined.
\end{thmAH1}

\newtheorem{thmAH2}{Theorem}
\begin{thmAH2}
	$C^m(\bar{I}) \subset W^m(I)$ and if $u \in C^m(C^m(\bar{I})C^m(\bar{I})\bar{I})$, then $D^{(m)}u = u^{(m)}$. 
\end{thmAH2}

\section*{Appendix: Mathematical Preliminaries}
In this appendix, we provide a brief overview of the mathematical spaces and concepts used throughout the dissertation. 

\subsection*{The Space \( L^2 \)}
Consider a measurable space \( X \). The set of square-integrable functions over \( X \) with respect to a measure \( \mu \) is called the \( L^2 \) space, formally defined as
\[ L^2(X, \mu) = \{ f: X \to R \mid \int_X |f|^2 \, d\mu < \infty \}. \]

The inner product in \( L^2 \) space is defined as
\[ (f,g) = \int_X f(x)g(x) \, d\mu \quad \text{for } f, g \in L^2(X, \mu). \]
The norm in \( L^2 \) is given by \( ||f|| = (f,f)^{1/2} \) for \( f \in L^2(X, \mu) \). For a comprehensive treatment, see \cite{Rud53}.

\textcolor{red}{***********************************************}

\subsection*{Spaces of Continuous Functions}

\textbf{The Space \( C^m(a,b) \)}\\
This space consists of functions that are \( m \)-times continuously differentiable in the open interval \( (a, b) \).

\textbf{The Space \( C^m[a,b] \)}\\
Functions in \( C^m[a,b] \) are \( m \)-times continuously differentiable in \( (a, b) \) and have defined right derivatives at \( a \) and left derivatives at \( b \) up to the \( m \)-th order.

\textbf{The Space \( C_0^m(a,b) \)}\\
The space \( C_0^m(a,b) \) contains functions from \( C^m[a,b] \) that vanish on the intervals \( [a, \alpha] \) and \( [\beta, b] \) for some \( a < \alpha < \beta < b \).

\textbf{The Spaces \( C^\infty(a,b) \) and \( C^\infty[a,b] \)}\\
These spaces contain functions that are infinitely differentiable, either on the open interval \( (a, b) \) or on the closed interval \( [a, b] \), respectively.

\textbf{The Space \( C_0^\infty(a,b) \)}\\
It includes all functions from \( C^\infty(a,b) \) that have compact support within \( (a, b) \).

\subsection*{Weak Derivatives and Sobolev Spaces}
Weak derivatives allow us to handle functions that are not classically differentiable. These derivatives are defined in a weak sense through the use of test functions.

\textbf{Definition: First Order Weak Derivative}\\
A function \( v \) is the first order weak derivative of \( u \in L^2(a,b) \) if
\[ (u, \phi') = -(v, \phi) \quad \text{for all } \phi \in C^\infty_0(a,b). \]

\textbf{Definition: Higher Order Weak Derivative}\\
Similarly, \( v \) is the \( m \)-th order weak derivative of \( u \) if
\[ (u, \phi^{(m)}) = (-1)^m(v, \phi) \quad \text{for all } \phi \in C^\infty_0(a,b). \]

\textbf{Sobolev Spaces \( W^m(a,b) \)}\\
The Sobolev space \( W^m(a,b) \) includes functions in \( L^2(a,b) \) with weak derivatives up to order \( m \). The inner product and norm in \( W^m(a,b) \) are defined as
\[ (u,v)_m = (u,v) + (Du, Dv) + \dots + (D^m u, D^m v), \]
\[ ||u||_m = \sqrt{(u,u)_m}. \]

\textbf{Sobolev Spaces \( H^m(a,b) \)}\\
The space \( H^m(a,b) \) is defined as the closure of \( C^m[a,b] \) in \( W^m(a,b) \) with respect to the Sobolev norm \( ||\cdot||_m \).

\subsection*{Main Results}
\textbf{Uniqueness of Weak Derivatives}\\
\begin{thmAH1}
	The \( m \)-th order weak derivative \( D^{(m)}u \) is uniquely determined by \( u \).
\end{thmAH1}

\textbf{Inclusion of Continuously Differentiable Functions}\\
\begin{thmAH2}
	The space \( C^m(\overline{I}) \) is a subset of \( W^m(I) \). For a function \( u \) in \( C^m(\overline{I}) \), the classical and weak derivatives coincide: \( D^{(m)}u = u^{(m)} \).
\end{thmAH2}

% Include more theorems, lemmas, or corollaries as needed
% ...

\end{document}
