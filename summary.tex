\documentclass[main.tex]{subfiles}
\begin{document}

\textbf{Title} Finite element analysis of multi-dimensional and simplified models for beams and plates

\textbf{Name} Rudi du Toit

\textbf{Supervisor} Prof NFJ van Rensburg

\textbf{Department} Mathematics and Applied Mathematics

\textbf{Degree} MSc Applied Mathematics

\begin{center}
	\large \textbf{Summary}
\end{center}
This dissertation is a literature study, investigating the Finite Element Analysis of multidimensional and simplified beam and plate models. The Timoshenko beam and Reissner-Mindlin plate theories are popular theories used in mathematics and engineering. An example of ongoing research using these theories is the study of suspension bridges. The articles \cite{Mck99}, \cite{BOC20} are examples of mathematical studies while \cite{SV86} follows an engineering approach.\\


 This dissertation is a literature study on the Finite Element Analysis 
 \begin{itemize}
	\item Considering linear mathematical model s for elastic beams and plates.
	\item Focus is on the one-dimensional Timoshenko beam theory and two-dimensional Reissner-Mindlin plate theory
	\item Also require two and three-dimensional beam models and a three-dimensional plate model
	\item Models are used in variational form (Chapter 1)
	\item Existence and uniqueness of solutions (Chapter 2)
	\item Convergence of FEM (Chapter 3)
	\item Chapter 4: - Timoshenko beam theory - Modal analysis - Examples of calculating eigenvalues and eigenvectors - Emperical study - Tipbody model with numerical results.
	\item Chapter 5: FEM applied to cantilever 3D, 2D and plate elastic bodies
	\item Chapter 6: Validity of Timo and Reissner-Mindlin Models
\end{itemize}


\end{document}
