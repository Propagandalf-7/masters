\documentclass[../main.tex]{subfiles}
\begin{document}
\section{Overview}
\begin{itemize}
	\item Chapter 1: Models, Dimensionless, Variational Forms, Modal Analysis
	\item Chapter 2: Example Weak Variational, Existence and Uniqueness from [VV02], Application to example
	\item Chapter 3: Example Galerkin, Convergence of Galerkin [BV13], Application to example, Convergence of Eigenvalues using FEM
	\item Chapter 4: Eigenvalues of Timo [VV06], Example of applications on models, Emperical Study of validity [SP06], Cantilever with Tip-Body (Example why need validity)
	\item Chapter 5: FEM of 2D, 3D, and Plate
	\item Chapter 6: Validity of 1D, Validity of 2D, Validity of Plate
	\item Chapter 7: Conclusion
\end{itemize}


In chapter 1, word al die modelle gegee wat in die dissertation gebruik word. Die modelle word in dimenielose form herskryf. Die modelle is dan herskryd in variational form wat gebruik gaan word vir die res van die verhandeling. Modal analysis word ook in chapter 1 verduidelik in hand vn 'n voorbeeld.\\

 In Chapter 2 word die werk van die artikel [VV02] gebruik om na die existence and uniqueness of solutions van second order hyperbolic equations te kyk. Die cantilever beam word as 'n voorbeeld gebruik om die werk te verduidelik.\\
 
 Chapter 3 kyk na die convergence van die FEM. Dit volg die werk van die artikel [XX]. Die methode van die artikel is om error estimates vir die fully discrete vartiational problem (Galerkin Approximation) te bepaal. In die tweede deel van chapter 3, word daar na die convergence van die eigenvalues en eigenvectors in die FEM gekyk. Dit volg die werk van die handboek [XX] en word in beter notation oorgeskryf.\\
 
 Chapter 4 fokus op the Timoshenko beam. Eerste word daar gekyk na die eigenvalue problem van die Timoshenko beam theory. Die artikel [VV06] gee die nodige werk om die eigenvalues en eigenvectors exactly uit te werk. Die methode word dan op 'n free-free beam en ook 'n cantilever beam toe gepas waar die eiewaarde uitgewerk word en ook die eigenfunctions geskets word. Daar is dan navorsing gedoen deur [SP06] wat gekyk het hoe goed 'n Timoshenko beam theory vergelyk met 'n 3D en 'n regte balk. Die navorsing wys dat die Timoshenko goed vergelyk die die regte balk in 'n free-free configuration. Die navorsing van [SP06] word discuss en gebruik dit as motiveering vir verdere numeriese simulation. Laaste deel in chapter 4 is 'n voorbeeld van 'n Timoshenko beam in 'n model wat meer real world application het. Dit is in die tegniese report van [XX]. Die werk in die chapter oor die tip body balk is net gefokus op die numeriese simulation.\\
 
 Chapter 5 gaan oor die plaat model. Die chapter doen die FEM analiese van die cantilever model. Die chapter le oook die reference configuration uit vir 'n rectangular plate model en adress die shear correction factor. Kan moontlik deel word van die volgende chapter.\\
 
 Chapter 6 is dieselfde as die vorige chapter Die chapter hanteer die FEM analise vir die twee en drie dimensionele elastic bodies. Die reference configuration is vir 'n rectangular form vir beide sodat dit later gebruik kan word indie volgende chapter. Die chapter gee ook die eigenvalue problems en wys wys dat die regte accuracy kan obtained word die die eiewaardes numeries uit te werk.\\
 
 Die laaste chapter hanteer die hoof numeriese werk van die verhandeling. Die die chapter word die validityvna die timoshenko beam asook die plaat model ondersoek. Eerste word die timoshenko beam met 'n twee-D beam vergelyk. Die is dieselfde as die artikel [XX]. So dieselfde werk word gedoen as die artikel, maar met aner waardes. Die werk word dan extend deur die twee-D model te vergelyk met die 3D model. Die modelle is elastic bodies, so daar work ook apart na die beam-type en non-beam type dele gekyk. Die eiewaardes word vergelyk met verskillende diktes va die 3D model en dan word daar gekyk na die eiefunksies se forms. Dan 
\end{document}
