\documentclass[../main.tex]{subfiles}
\begin{document}


\section{Comparison of models}
\subsection{Model problem for a three-dimensional elastic solid}

Models for a three-dimensional elastic solid is presented. These models are a cantilever and a free-free elastic body. A dimensionless variational form of the cantilever model is derived. General plane stress is discussed.

\subsection{Two-dimensional model problem for an elastic solid}

A model for a two-dimensional elastic solid is presented. The model is a cantilever elastic body. This model is a special case of plane stress of a three-dimensional elastic body. The model is given in a dimensionless variational form by simplification of the three-dimensional model.

\subsection{Timoshenko beam models}

Models for a one-dimensional beam are presented using the Timoshenko beam theory. Four different models are presented. The models are a cantilever, free-free, pinned-pinned and a suspended beam. A dimensionless variational form for the models are derived.

\subsection{Reissner-Mindlin Plate Model}

Models for a Reissner-Mindlin plate are presented. The models are a cantilever and a rigidly clamped plate. A dimensionless variational form for the cantilever model is derived.

\subsection{Rod Theory}
**************

\section{Mathematical analysis of vibration problems}

This chapter is a discussion and example of application of the theory presented in \cite{VV02}. This article concerns the existence and uniqueness of solutions of second order hyperbolic type problems. Like the models in this dissertation.\\

As an example to explain the theory, the a weak variational form of a cantilever Timoshenko beam is used.\\

This chapter also concerns the idea of modal analysis. This is a method to find the natural frequencies of a system. This method is used later in the dissertation.

\subsection{Existence and uniqueness of solutions}

\subsubsection{The variational approach}

A general variational problem is defined. Specific Hilbert spaces are defined as well as notation. The four assumptions from \cite{VV02} is given. 

\subsubsection{Main results for existence and uniqueness}
The main results from \cite{VV02} is presented. This includes cases for strong and weak damped systems. The models in this dissertation all have weak damping.

\subsubsection{First order system}
The general variational problem is rewritten for a first order system. To obtain this first order system, a operator $A$ is defined. The authors prove that this operator is a infinitesimal generator of a $C_0$-semigroup of contractions with a domain in the Hilbert space $H$.\\

The existence and uniqueness results for the first order system is easily proven with an initial value problem. The authors also prove that a solution of the first order system is also a solution to the general variational problem.

\subsection{Application: Timoshenko beam model}
The theory is applied to a cantilever Timoshenko beam model. Using the weak variational form, the four assumptions from \cite{VV02} is proven.

\subsection{Modal analysis}
*****************************************************************

\subsubsection{Timoshenko beam}
\subsubsection{General vibration problem}
\section{Finite element theory}

\subsection{Convergence of the Galerkin approximation}
In this chapter, two convergence results relating to the Finite Element Method is presented. The first result is a convergence result for the Galerkin approximation of a second order hyperbolic type problem. This is presented in \cite{BV13}.\\

The second result is from the textbook \cite{SF97}. This results concerns the convergence of the eigenvalues and eigenfunctions of a one-dimensional second order hyperbolic type problem when using the Finite Element Method.

\subsubsection{Galerkin approximation}
The general variational problem is identical to the one in \cite{VV02} presented in chapter 2. It is assumed that the the general variational problem has a unique solution.\\

The Galerkin approximation for the general variational problem is derived. Uniqueness of a solution to the Galerkin approximation is proven if $f$ is continuous.\\


\subsubsection{System of ordinary differential equations}

Continuing with to apply the Finite Element Method, the Galerkin approximation is rewritten as a system of ordinary differential equations. The system is a linear system.\\

\subsubsection{Error estimates}
Suppose $u$ is a solution to the general variational problem and $u^h$ is a solution to the Galerkin approximation. The goal of \cite{BV13} is to prove that $u^h$ converges to $u$ in $H$ as $h\rightarrow 0$.\\

A error estimate is defined by the authors. The error estimate is split into two parts. A semi-discrete part and a fully discrete part. The two parts are investigated separately.\\

\subsubsection{Main result}

The main result of \cite{BV13} is presented. Two assumptions of \cite{BV13} are also given.

\subsubsection{Obtaining the error estimate}

In this subsection, the approach of the authors are discussed. First a error estimate for the semi-discrete problem is obtained.*****\\

\subsection{Convergence of eigenvalues in Finite Element Analysis}
*****************************************************************

\subsubsection{Eigenvalue problem for vibration problem}
\textbf{Problem GVarE}
\subsubsection{Interpolation theorem}
\subsubsection{Eigenvalue problem for our models*********}
\textbf{Notation}
\subsubsection{Convergence of the eigenvalues}
\subsubsection{Convergence of the eigenfunctions}
\subsection{Example}
\textbf{Problem 2D-1V}\\
\textbf{Test functions}
\subsubsection{Weak variational form}
\textbf{Problem 2D-1W}
\section{Timoshenko beam model}

In this chapter, the Timoshenko beam theory is explored further. 


\subsection{Eigenvalue problem}
In the first three sections, the idea of modal analysis is applied to the Timoshenko beam theory. The first section follows the work of \cite{VV06}. In this article, the authors present a method of calculating the eigenvalues and eigenfunctions. 
\textbf{General eigenvalue problem}\\
\textbf{Case $\bm {\lambda <\alpha }$}\\
\textbf{Case $\bm { \lambda =\alpha }$}\\
\textbf{Case $\bm {\lambda >\alpha }$}
\subsection{Cantilever beam}
This is an example of the application of the theory on a cantilever Timoshenko beam.
\subsubsection{Calculating the eigenvalues}
\subsubsection{Example of mode shapes}
\subsection{Free-free timoshenko beam}

This is an example of the application of the theory on a free-free Timoshenko beam.

\subsubsection{Calculating the eigenvalues}
\subsubsection{Example of mode shapes}
\subsection{Suspended beam model}
The idea of a suspended Timoshenko is also discuss briefly.\\
\textbf{Variational problem}
\subsection{SP06}
In the next section, an empirical study of the Timoshenko beam is discussed. In the article \cite{SP06} the authors obtained the eigenvalues of a Timoshenko beam, by analyzing the natural frequencies of a real world beam. Their findings are compared to various theoretical models.\\
\subsubsection{Mathematical models}
\textbf{Problem 3D-2}
\subsubsection{Experimental setup}
\subsubsection{Results from SP06}
\subsection{Cantilever Timoshenko beam with tip-body}
Last an example of the Timoshenko beam in application is presented. This section is a study of a technical report on a cantilever Timoshenko beam with elastic interfaces and a tip body.
\subsubsection{The model problem}
\textbf{Boundary conditions}\\
\textbf{Remark}
\subsubsection{Variational form}
\textbf{Remark}\\
\textbf{Interface equation}\\
\textbf{Problem Var}
\subsubsection{Galerkin approximation}
\textbf{Problem G}
\subsubsection{System of ordinary differential equations}
\textbf{Problem ODE}
\subsubsection{Numerical results}
\textbf{Case: Control}\\
\textbf{Case: Decrease $\gamma $}\\
\textbf{Case: Decrease $\mu $}\\
\textbf{Case: Increase $m$}
\section{Finite element method}
\subsection{Two-dimensional cantilever body}
\textbf{Reference configuration for rectangular cross-section}\\
\textbf{Cantilever elastic body}\\
\textbf{Problem 2D-1V}\\
\textbf{Constitutive equations:}\\
\textbf{Bilinear form:}
\subsubsection{Weak variational form}
\textbf{Problem 2D-1WV}
\subsubsection{Galerkin approximation}
\textbf{Problem 2D-1G}
\subsubsection{System of differential equations}
\textbf{FEM matrices}\\
\textbf{Problem 2D-1ODE}\\
\textbf{Remark}
\subsubsection{Eigenvalue problem}
\textbf{Problem 2D-1E}
\subsection{Three-dimensional elastic model}
\textbf{Reference configuration for rectangular cross-section}\\
\textbf{Cantilever elastic body}\\
\textbf{Problem 3D-1V}\\
\textbf{Bilinear Form}
\subsubsection{Weak variational form}
\textbf{Bilinear form}\\
\textbf{Problem 3D-1W}
\subsubsection{Galerkin approximation}
\textbf{Problem 3D-1G}
\subsubsection{System of ordinary differential equations}
\textbf{FEM matrices}\\
\textbf{Problem 3D-1ODE}
\subsubsection{Eigenvalue problem}
\textbf{Problem 3D-1E}
\subsection{Cantilever plate model}
\textbf{Reference configuration for a rectangular plate}\\
\textbf{Cantilever plate model}\\
\textbf{Problem P-1V}\\
\textbf{Constitutive Equations}\\
\textbf{Bilinear Form}
\subsubsection{Weak variational form}
\textbf{Bilinear forms}\\
\textbf{Problem Plate-1W}\\
\textbf{Existence and uniqueness}
\subsubsection{Galerkin approximation}
\textbf{Problem P-1G}
\subsubsection{System of ordinary differential equations}
\textbf{FEM matrices}\\
\textbf{Problem P-1ODE}
\subsubsection{Eigenvalue problem}
\textbf{Problem P-1E}
\section{Validity of cantilever beam and plate models}
\textbf{Parameters}
\subsection{Validity of the Timoshenko beam model}
\subsubsection{The models}
\subsubsection{Accuracy of the numerical eigenvalues}
\subsubsection{Comparing the shape of the eigenfunctions}
\textbf{Shapes relating to beam-type eigenvalues}\\
\textbf{Shapes relating to non-beam type eigenvalues}\\
\textbf{Shapes of the cross-section}\\
\textbf{Direct comparison of mode shapes}\\
\textbf{Remark}
\subsubsection{Comparing the eigenvalues}
\textbf{Results of [LVV09]}
\subsubsection{Comparison of the eigenvalues}
\subsection{Validity of the two-dimensional elastic model}
\subsubsection{The models}
\subsubsection{Accuracy of the numerical eigenvalues}
\subsubsection{Comparing the mode shapes}
\textbf{Mode shapes relating to beam type eigenvalues.}\\
\textbf{Mode shapes relating to non-beam type eigenvalues that are present in the two-dimensional model.}\\
\textbf{Mode shapes relating to non-beam type eigenvalues that are not present in the two-dimensional model.}
\subsubsection{Comparing the eigenvalues}
\subsubsection{$b > h$}
\subsection{Validity of the Reissner-Mindlin plate model}
\subsubsection{Comparing the mode shapes}
\subsubsection{Comparing the eigenvalues}
\section{Overview}
\subsection{Overview}
\section{Conclusion}
\subsection{Overview}
\textbf{Properties}\\
\textbf{Properties of Type One}\\
\textbf{Properties of Type Two}\\
\textbf{Properties}
\textbf{Different Research Approaches}\\
\textbf{Motivation}
\subsection{NRF Research Proposal}
\subsection{Modelling a Suspension Bridge: A Brief History}
\subsection{Modelling a suspension bridge: An Engineering Example}
\subsection{Simplified Mathematical Model for a Suspended Beam}
\subsubsection{Reference Configuration}
\subsubsection{System of Equations}
\textbf{Equations of Motion}\\
\textbf{Constitutive Equations}\\
\textbf{Dimensionless Equations}
\subsubsection{Boundary Conditions}
\subsection{Analysis of a Thermoelastic Timoshenko Beam Model}
\textbf{Dimensionless Equations}
\subsection{An Elastically Clamped Beam Model}
\textbf{Boundary Conditions}\\
\textbf{Hooke's law in two-dimensions}


******************************************************************\\

\section{Comparison of models}
\subsection{Model problem for a three-dimensional elastic solid}
\subsubsection{Equations of motion and constitutive equations}
\textbf{Equation of motion}
\subsubsection{Dimensionless form}
\subsubsection{Model problems}
\textbf{Problem 3D-1}\\
\textbf{Problem 3D-2}
\subsubsection{Variational form}
\textbf{Bilinear forms and integral}\\
\textbf{Test functions}\\
\textbf{Problem 3D-1V}
\subsubsection{Plane stress}
\subsection{Two-dimensional model problem for an elastic solid}
\subsubsection{Equations of motion and constitutive equations}
\textbf{Equation of motion}\\
\textbf{Constitutive equation}
\subsubsection{Model problem}
\textbf{Problem 2D-1}
\subsubsection{Variational form}
\textbf{Bilinear forms and integral}\\
\textbf{Test functions for problem 2D-1}\\
\textbf{Problem 2D-1V}
\subsection{Timoshenko beam models}
\subsubsection{Equations of motion and constitutive equations}
\subsubsection{Boundary conditions}
\subsubsection{Boundary value problems}
\subsubsection{Variational form}
\textbf{Function spaces}\\
\textbf{Test function spaces for different problems}\\
\textbf{Bilinear forms}
\subsection{Reissner-Mindlin Plate Model}
\subsubsection{Equations of Motion and Constitutive Equations}
\subsubsection{Dimensionless Form}
\textbf{Remark}\\
\textbf{Dimensionless Equations of Motion}\\
\textbf{Dimensionless Constitutive Equations}
\subsubsection{Boundary Conditions}
\subsubsection{Model Problems}
\textbf{Problem P-1}\\
\textbf{Problem P-2}
\subsubsection{Variational Form}
\textbf{Bilinear Forms and integral}\\
\textbf{Test Functions for Problem P-1}\\
\textbf{Problem P-1V}
\subsection{Rod Theory}
\section{Mathematical analysis of vibration problems}
\textbf{Problem T-2W}
\subsection{Existence and uniqueness of solutions}
\subsubsection{The variational approach}
\subsubsection{Main results for existence and uniqueness}
\subsubsection{First order system}
\subsection{Application: Timoshenko beam model}
\subsection{Modal analysis}
\subsubsection{Timoshenko beam}
\subsubsection{General vibration problem}
\section{Finite element theory}
\subsection{Convergence of the Galerkin approximation}
\subsubsection{Galerkin approximation}
\subsubsection{System of ordinary differential equations}
\subsubsection{Error estimates}
\subsubsection{Main result}
\subsubsection{Obtaining the error estimate}
\textbf{Semi-discrete problem}
\subsubsection{Fully discrete Galerkin finite element approximation}
\textbf{Notation}\\
\textbf{Error estimate for the fully discrete problem}
\subsection{Convergence of eigenvalues in Finite Element Analysis}
\subsubsection{Eigenvalue problem for vibration problem}
\textbf{Problem GVarE}
\subsubsection{Interpolation theorem}
\subsubsection{Eigenvalue problem for our models*********}
\textbf{Notation}
\subsubsection{Convergence of the eigenvalues}
\subsubsection{Convergence of the eigenfunctions}
\subsection{Example}
\textbf{Problem 2D-1V}\\
\textbf{Test functions}
\subsubsection{Weak variational form}
\textbf{Problem 2D-1W}
\section{Timoshenko beam model}
\subsection{Eigenvalue problem}
\textbf{General eigenvalue problem}\\
\textbf{Case $\bm {\lambda <\alpha }$}\\
\textbf{Case $\bm { \lambda =\alpha }$}\\
\textbf{Case $\bm {\lambda >\alpha }$}
\subsection{Cantilever beam}
\subsubsection{Calculating the eigenvalues}
\subsubsection{Example of mode shapes}
\subsection{Free-free timoshenko beam}
\subsubsection{Calculating the eigenvalues}
\subsubsection{Example of mode shapes}
\subsection{Suspended beam model}
\textbf{Variational problem}
\subsection{SP06}
\subsubsection{Mathematical models}
\textbf{Problem 3D-2}
\subsubsection{Experimental setup}
\subsubsection{Results from SP06}
\subsection{Cantilever Timoshenko beam with tip-body}
\subsubsection{The model problem}
\textbf{Boundary conditions}\\
\textbf{Remark}
\subsubsection{Variational form}
\textbf{Remark}\\
\textbf{Interface equation}\\
\textbf{Problem Var}
\subsubsection{Galerkin approximation}
\textbf{Problem G}
\subsubsection{System of ordinary differential equations}
\textbf{Problem ODE}
\subsubsection{Numerical results}
\textbf{Case: Control}\\
\textbf{Case: Decrease $\gamma $}\\
\textbf{Case: Decrease $\mu $}\\
\textbf{Case: Increase $m$}
\section{Finite element method}
\subsection{Two-dimensional cantilever body}
\textbf{Reference configuration for rectangular cross-section}\\
\textbf{Cantilever elastic body}\\
\textbf{Problem 2D-1V}\\
\textbf{Constitutive equations:}\\
\textbf{Bilinear form:}
\subsubsection{Weak variational form}
\textbf{Problem 2D-1WV}
\subsubsection{Galerkin approximation}
\textbf{Problem 2D-1G}
\subsubsection{System of differential equations}
\textbf{FEM matrices}\\
\textbf{Problem 2D-1ODE}\\
\textbf{Remark}
\subsubsection{Eigenvalue problem}
\textbf{Problem 2D-1E}
\subsection{Three-dimensional elastic model}
\textbf{Reference configuration for rectangular cross-section}\\
\textbf{Cantilever elastic body}\\
\textbf{Problem 3D-1V}\\
\textbf{Bilinear Form}
\subsubsection{Weak variational form}
\textbf{Bilinear form}\\
\textbf{Problem 3D-1W}
\subsubsection{Galerkin approximation}
\textbf{Problem 3D-1G}
\subsubsection{System of ordinary differential equations}
\textbf{FEM matrices}\\
\textbf{Problem 3D-1ODE}
\subsubsection{Eigenvalue problem}
\textbf{Problem 3D-1E}
\subsection{Cantilever plate model}
\textbf{Reference configuration for a rectangular plate}\\
\textbf{Cantilever plate model}\\
\textbf{Problem P-1V}\\
\textbf{Constitutive Equations}\\
\textbf{Bilinear Form}
\subsubsection{Weak variational form}
\textbf{Bilinear forms}\\
\textbf{Problem Plate-1W}\\
\textbf{Existence and uniqueness}
\subsubsection{Galerkin approximation}
\textbf{Problem P-1G}
\subsubsection{System of ordinary differential equations}
\textbf{FEM matrices}\\
\textbf{Problem P-1ODE}
\subsubsection{Eigenvalue problem}
\textbf{Problem P-1E}
\section{Validity of cantilever beam and plate models}
\textbf{Parameters}
\subsection{Validity of the Timoshenko beam model}
\subsubsection{The models}
\subsubsection{Accuracy of the numerical eigenvalues}
\subsubsection{Comparing the shape of the eigenfunctions}
\textbf{Shapes relating to beam-type eigenvalues}\\
\textbf{Shapes relating to non-beam type eigenvalues}\\
\textbf{Shapes of the cross-section}\\
\textbf{Direct comparison of mode shapes}\\
\textbf{Remark}
\subsubsection{Comparing the eigenvalues}
\textbf{Results of [LVV09]}
\subsubsection{Comparison of the eigenvalues}
\subsection{Validity of the two-dimensional elastic model}
\subsubsection{The models}
\subsubsection{Accuracy of the numerical eigenvalues}
\subsubsection{Comparing the mode shapes}
\textbf{Mode shapes relating to beam type eigenvalues.}\\
\textbf{Mode shapes relating to non-beam type eigenvalues that are present in the two-dimensional model.}\\
\textbf{Mode shapes relating to non-beam type eigenvalues that are not present in the two-dimensional model.}
\subsubsection{Comparing the eigenvalues}
\subsubsection{$b > h$}
\subsection{Validity of the Reissner-Mindlin plate model}
\subsubsection{Comparing the mode shapes}
\subsubsection{Comparing the eigenvalues}
\section{Overview}
\subsection{Overview}
\section{Conclusion}
\subsection{Overview}
\textbf{Properties}\\
\textbf{Properties of Type One}\\
\textbf{Properties of Type Two}\\
\textbf{Properties}
\textbf{Different Research Approaches}\\
\textbf{Motivation}
\subsection{NRF Research Proposal}
\subsection{Modelling a Suspension Bridge: A Brief History}
\subsection{Modelling a suspension bridge: An Engineering Example}
\subsection{Simplified Mathematical Model for a Suspended Beam}
\subsubsection{Reference Configuration}
\subsubsection{System of Equations}
\textbf{Equations of Motion}\\
\textbf{Constitutive Equations}\\
\textbf{Dimensionless Equations}
\subsubsection{Boundary Conditions}
\subsection{Analysis of a Thermoelastic Timoshenko Beam Model}
\textbf{Dimensionless Equations}
\subsection{An Elastically Clamped Beam Model}
\textbf{Boundary Conditions}\\
\textbf{Hooke's law in two-dimensions}








\begin{itemize}
	\item Chapter 1: Models, Dimensionless, Variational Forms, Modal Analysis
	\item Chapter 2: Example Weak Variational, Existence and Uniqueness from [VV02], Application to example
	\item Chapter 3: Example Galerkin, Convergence of Galerkin [BV13], Application to example, Convergence of Eigenvalues using FEM
	\item Chapter 4: Eigenvalues of Timo [VV06], Example of applications on models, Emperical Study of validity [SP06], Cantilever with Tip-Body (Example why need validity)
	\item Chapter 5: FEM of 2D, 3D, and Plate
	\item Chapter 6: Validity of 1D, Validity of 2D, Validity of Plate
	\item Chapter 7: Conclusion
\end{itemize}


In chapter 1, word al die modelle gegee wat in die dissertation gebruik word. Die modelle word in dimenielose form herskryf. Die modelle is dan herskryd in variational form wat gebruik gaan word vir die res van die verhandeling. Modal analysis word ook in chapter 1 verduidelik in hand vn 'n voorbeeld.\\

 In Chapter 2 word die werk van die artikel [VV02] gebruik om na die existence and uniqueness of solutions van second order hyperbolic equations te kyk. Die cantilever beam word as 'n voorbeeld gebruik om die werk te verduidelik.\\
 
 Chapter 3 kyk na die convergence van die FEM. Dit volg die werk van die artikel [XX]. Die methode van die artikel is om error estimates vir die fully discrete vartiational problem (Galerkin Approximation) te bepaal. In die tweede deel van chapter 3, word daar na die convergence van die eigenvalues en eigenvectors in die FEM gekyk. Dit volg die werk van die handboek [XX] en word in beter notation oorgeskryf.\\
 
 Chapter 4 fokus op the Timoshenko beam. Eerste word daar gekyk na die eigenvalue problem van die Timoshenko beam theory. Die artikel [VV06] gee die nodige werk om die eigenvalues en eigenvectors exactly uit te werk. Die methode word dan op 'n free-free beam en ook 'n cantilever beam toe gepas waar die eiewaarde uitgewerk word en ook die eigenfunctions geskets word. Daar is dan navorsing gedoen deur [SP06] wat gekyk het hoe goed 'n Timoshenko beam theory vergelyk met 'n 3D en 'n regte balk. Die navorsing wys dat die Timoshenko goed vergelyk die die regte balk in 'n free-free configuration. Die navorsing van [SP06] word discuss en gebruik dit as motiveering vir verdere numeriese simulation. Laaste deel in chapter 4 is 'n voorbeeld van 'n Timoshenko beam in 'n model wat meer real world application het. Dit is in die tegniese report van [XX]. Die werk in die chapter oor die tip body balk is net gefokus op die numeriese simulation.\\
 
 Chapter 5 gaan oor die plaat model. Die chapter doen die FEM analiese van die cantilever model. Die chapter le oook die reference configuration uit vir 'n rectangular plate model en adress die shear correction factor. Kan moontlik deel word van die volgende chapter.\\
 
 Chapter 6 is dieselfde as die vorige chapter Die chapter hanteer die FEM analise vir die twee en drie dimensionele elastic bodies. Die reference configuration is vir 'n rectangular form vir beide sodat dit later gebruik kan word indie volgende chapter. Die chapter gee ook die eigenvalue problems en wys wys dat die regte accuracy kan obtained word die die eiewaardes numeries uit te werk.\\
 
 Die laaste chapter hanteer die hoof numeriese werk van die verhandeling. Die die chapter word die validityvna die timoshenko beam asook die plaat model ondersoek. Eerste word die timoshenko beam met 'n twee-D beam vergelyk. Die is dieselfde as die artikel [XX]. So dieselfde werk word gedoen as die artikel, maar met aner waardes. Die werk word dan extend deur die twee-D model te vergelyk met die 3D model. Die modelle is elastic bodies, so daar work ook apart na die beam-type en non-beam type dele gekyk. Die eiewaardes word vergelyk met verskillende diktes va die 3D model en dan word daar gekyk na die eiefunksies se forms. Dan 
\end{document}
