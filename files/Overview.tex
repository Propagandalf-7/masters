\documentclass[../main.tex]{subfiles}
\begin{document}

\section{500-word summary}
This dissertation is a literature study that investigates the validity of different linear models in application. Validity in this context refers to how well simplified lower dimensional models compare to more realistic higher dimensional models. This dissertation will investigate the validity of the Timoshenko beam theory and the Reissner-Mindlin plate theory. The higher dimensional models are a two- and three-dimensional beam and a three-dimensional plate. In the first chapter, the models are provided.

Next, a discussion of the existence and uniqueness of solutions for a general vibration problem. The article discussed provide four assumptions, and prove there exists an infinitesimal generator of a $C_0$-semigroup of contractions, that rewrites the problem into an initial value problem, with a unique solution. The authors then prove this is also a solution of the general vibration problem. An application of this theory is demonstrated using a cantilever Timoshenko beam.

Thereafter a discussion of two theoretical results for the Finite Element Method. The first result is from an article where the convergence of the Galerkin Approximation is proved. A Galerkin Approximation of the general vibration problem is derived. Using the Finite Element Method, this Galerkin Approximation is rewritten into an initial value problem, that has a unique solution. The authors show that there exists error estimates that approximate the error between the general vibration problem and the initial value problem. The next result is from a textbook on the convergence of the eigenvalues and eigenvectors of a one-dimensional vibration problem. This section rewrites the textbook's results with updated notation, aiming for greater clarity and understanding.

The forth chapter discusses results for the Timoshenko beam theory. First an article is examined that provides a method to calculate the exact eigenvalues and eigenvectors. The method is then applied to a cantilever and a free-free beam model. Next is a discussion of an empirical study, that compares the natural frequencies of a physical free-free beam to the theoretical results of the Timoshenko beam theory. Finally an extension on a technical report of a cantilever Timoshenko beam with a tip-body and elastic interfaces. This model serves as an example of the application of the Timoshenko theory in a complex model and adds to the article by expanding the numerical results.

In the fifth chapter, the Finite Element Method is applied to the two and three-dimensional models of this dissertation. The goal of this chapter is to obtain a numerical formula that can be used to calculate the eigenvalues and eigenvectors of the models in preparation for the final chapter where the models are compared.

The last chapter compares the eigenvalues and eigenvectors of the models. The approach is based on an article where a cantilever Timoshenko beam model is compared to a cantilever two-dimensional model. The article is discussed and results replicated. The article is then extended to a comparison of a two-dimensional model to a three-dimensional model. Finally, a cantilever Reissner-Mindlin plate model is compared to a cantilever three-dimensional plate model.

\textcolor{red}{493 Words}

\section{Summary}

\subsubsection{Chapter 1}
This dissertation is a literature study that investigates the validity of different linear models. The term validity in this dissertation means how well a model compares to a more realistic model for real world applications. The first chapter of the dissertation introduces the models. The main model is the Timoshenko beam theory. The other models are a two-dimensional elastic model, a three-dimensional elastic model and a Reissner-Mindlin plate model. The goal of this disseration to validate the use of the Timoshenko beam model and the Reissner-Mindlin plate model in applications. These models are simplified one-dimensional beam and two-dimensional plate models. But more realistic models exists, such as a two-dimensional and three-dimensional beam model and a three-dimensional plate model. Using modal analysis, it is shown that the solutions of the models can be represented as a linear combination of the eigenvalues and eigenfunctions. Therefore it is only required to compare the eigenvalues and eigenfunctions of the models to determine the difference of the solutions. 

\subsubsection{Chapter 2}
In this section, the existence and solutions for a general vibration problem is investigated. The main article that is discussed is \cite{VV02}. In this article, the authors define a general vibration problem. The authors state four assumptions that need to be satified. Under these four assumptions, it can be shown that for the general vibration problem there exists a linear operator A, which is an infinitesimal generator of a $C_0$-semigroup of contractions with a domain in the problem space. This operator allows the general vibration problem to be rewritten as a system of first order differential equations $\displaystyle x' = Ax + f$. This differential equations are rewritten into an initial value problem which can be proven using semi-group theory to have a unique solution. The authors then provide the necessary proofs to show that a solution of the initial value problem is also a solution of the general vibration problem. An example of the application of this theory is provided using a cantilever Timoshenko beam.

\subsubsection{Chapter 3}
In this chapter, some theory for the Finite Element Method is discussed. This chapter contains two parts, that covers two different results. The first result is the convergence of the Galerkin approximation of a general vibration problem. This result is presented in the article \cite{BV13}. The second result concerns the convergence of the eigenvalues and eigenfunctions of a one-dimensional vibration problem when applying the Finite Element Method. This result is presented in the textbook \cite{SF97}. In the first section, the authors continue with the general vibration problem that is used in chapter 2. The general vibration problem has a solution as shown in chapter 2. The Galerkin Approximation is derived from the general vibration probelm and is rewritten into a system of ordinary differential using the Finite Element Method. This oridinary differential equation can be proven to have a unique solution. The main results of \cite{BV13} shows that this solution of the Galerkin Approximation converges to the solution of the general vibration problem. The approach of the authors is to calculate the error estimates. The error is split into two parts, a semi-discrete and a fully discrete problem. Two assumptions are added to the assumptions from chapter 2. Using these assumptions, the authors obtain error estimates for the semi-discrete and fully discrete problems.

The second part of the chapter considers work done in a textbook \cite{SF97}. The specific work discussed, covers the convergence of the eigenvalues and eigenfunctions of a one-dimensional vibration problem when applying the Finite Element Method. The authors consider a general eigenvalue problem. Then using the Rayleigh-quotient, from the Rayleigh-Ritz method, as well as an approximation theorem from \cite{ORXX}, the main result is proven. The specific work done in this section is updating the notation of the textbook, as well as expanding some results so that the results are easier to understand.

\subsubsection{Chapter 4}
This chapter is a focus on the main model theory of this disseration, the Timoshenko beam theory. The first section is a discussion of modal analysis applied to the Timoshenko beam theory, and specifically a discussion of the article \cite{VV06}. In this article, the authors present a method to calculate the exact eigenvalues and eigenfunctions of a Timoshenko beam. This method considers the boundary conditions, which differs from most other research. Starting with a general eigenvalue problem for a Timoshenko beam model, the authors derive a general solution for this problem. The authors then explain the method by hands of an example by applying the method to a cantilever beam model. The next sections of this chapter also then looks at examples of applying the method, first to a cantilever beam model, and then to a free-free beam model. The focus of these applications are to show that the method works and that the numerical results can be obtained easily. Similar numerical results will be required in chapter 6. The next section is a short discussion on a suspended beam model, afterwards an interesting article \cite{SP06} is discussed. In this article, the authors investigate the validity of the Timoshenko beam theory, by comparing the eigenvalues (natural frequencies) for a emperically mesasured beam, to a Timoshenko beam and a three-dimensional beam using Finite Element Analysis. The specific model used is a free-free beam. The emperical results are obtained by vibrating a suspended steel beam and measuring the natural frequencies. These results are then compared to the theoretical results. Finally the chapter finished by looking at an interesting model of a cantilever beam with a tip-body with elastic interfaces. These results combine modal analysis and looks at the behavior of the model. This section serves as an example of how useful the Timoshenko beam theory can be in a complex model.

\subsubsection{Chapter 5}
In this chapter, the Finite Element Method is applied to cantilever two-dimensional,and three-dimensional cantilever beams and a cantilever Reissner-Mindlin plate. The aim of this section is to obtain a numerical formula to calculate the eigenvalues and eigenfunctions of the models. The results of these formulas will be used in chapter 6. The Finite Element Method need not be applied to the Timoshenko beam theory, as chapter 4 provides an alternative method. For all the models, the Finite Element Method is applied using bi-cubic or tri-cubic basis functions to improve the rate of convergence and reduce the processing required to obtain accurate results.

\subsubsection{Chapter 6}
In this chapter, the main comparisons of the models are conducted. The first section is a discussion of the article \cite{LVV09}. Each section in this chapter follows the same structure as this article. In this article, the authors investigate the validity of a cantilever Timoshenko beam models, by comparing it to a two-dimensional cantilever Timoshenko beam model. To make this comparison, the authors compare the eigenvalues and eigenfunctions of the models. But since the two-dimensional model is more complex, there are `non-beam type' eigenvalues. These are eigenvalue that are not in the Timoshenko beam theory, and does not relate to beam type problems. To be able to find the beam type eigenvalues that can be compared, the authors compare the mode shapes. This allows the eigenvalues to be matched up. The results of the article show that the Timoshenko beam model compares very well to the two-dimensional beam. The comparison improves when the application is for a long and slender beam. The results of the article are verified and more results are added to extend the article and also to be using the the following sections. But in real-world applications, a beam is a three-dimensional model. Therefore it would be better to use a three-dimensional model to investigate the validity of the Timoshenko beam theory. The authors of \cite{LVV09} mentions this, and suggests using the two-dimensional model as an intermediate step to avoid complexities. So in the second section of this chapter, the validity of the two-dimensional model is investigated using the three-dimensional model as a reference. The results show that the two-dimensional model also compares very well to the three-dimensional model. Similar to the one-dimensional case, the results improve for a long and slender beam, and also if the width of the three-dimensional beam is not larger than the height of the beam. When increasing the width past the size of the height, the two-dimensional model is less accurate. This shape does however bring into question the use of a beam model, and something like a plate model might be better suited. In the last section, the validiy of the Reissner-Mindlin plate model is investigated. Similar to how the Timoshenko beam theory improves on the Euler-Bernoulli beam theory, the Reissner-Mindlin plate theory improves on the classical plate theory. The numerical results do show that this Reissner-Mindlin plate model compares well to a three-dimensional plate model, also when the dimensions of the models are realistic for application.



\end{document}

\section{Overview}

\subsubsection{Comparison of models}
The first chapter introduces the models used in this dissertation. The models are of different dimensions. The first model is a three dimensional elastic body. The model problems for the three-dimensional body is a cantilever and a free-free body. A general case of plane stress is also discussed. Making assumptions on plane stress, results into a two-dimensional model. This model is a special case of plane stress. For the one-dimensional case, the Timoshenko beam theory is considered. Four different models problems are used in this disseration. These model problems are a cantilever, a free-free beam, a pinned-pinned beam and a suspended beam. Lastly a Reissner Mindlin plate theory is considered. The model problems for the plate theory are a cantilever plate and a rigidly clamped plate. For each of the models problems, the boundary conditions are presented and the variational form is derived. The variational form of each problem is mainly used in this dissertation. (ROD THEORY?)

\subsubsection{Mathematical analysis of vibration problems}

This chapter is a discussion and example of application of the theory presented in the article \cite{VV02}. This article concerns the existence and uniqueness of solutions of second order hyperbolic type problems. The models in this dissertation are such problems. In the article \cite{VV02}, the authors define a general variational problem on specific Hilbert spaces. To obtain the main results of the article, four assumptions are made. The results include existence and uniqueness of solutions for strong and weak damped models. The models in this dissertation are all weakly damped. The general variational problem is rewritten for a first order system of differential equations. This system is defined on a Hilbert space $H$. A operator $A$ is defined, and using the assumptions the authors prove that the operator $A$ is an infinitesimal generator of a $C_0$-semigroup of contractions with a domain in $H$. Using this operator the authors rewrite the system of differential into the form $\displaystyle x' = Ax + f$. To be more precise, the system of first order system of differential equations can be written as an initial value problem. The existence and uniqueness results for the first order system is easily proven with an initial value problem. The authors also prove that a solution of the first order system is also a solution to the general variational problem. Using semi-group theory, a unique solution for the initial value problem can be proved. The authors also provide the link between the solutions of the initial value problem and the general variational form. Thus proving the required results. As an example to apply the theory, a cantilever Timoshenko beam is used. The weak variational form is derived. In this weak variational form, the product spaces can be made complete. The four assumptions of \cite{VV02} is then proved to hold for the example. (Modal Analysis ?)

\subsubsection{Finite element theory}

In this chapter, two convergence results relating to the Finite Element Method are discussed. The first result is the convergence of the Galerkin approximation for vibration problems. This is presented in the article \cite{BV13.} For this result, the general variational form of chapter 2 is used. The Galerkin approximation of this general variational problem can be derived, as well as an initial value problem using the standard Finite Element Analysis matrices. The article \cite{BV13} derive an error estimate for the solution of the general vartiational problem and the solution to it's Galerkin approximation. To obtain this error estimate, the authors make two assumptions, and split the problem into a semi-discrete and a fully discrete problem. Following the assumptions, the authors prove error estimates for the semi-dicrete and the fully-discrete problem.\\ 

The second result concerns the convergence of the eigenvalues and eigenfunctions with the Finite Element Method for one-dimensional vibration problems. This is a result form the textbook \cite{SF97}.**********


\subsubsection{Timoshenko beam model}

Tis chapter explores the idea of the Timoshenko beam theory further. The article \cite{VV06} presents a method of finding the exact eigenvalues and eigenfunctions for a Timoshenko beam. The method uses modal analysis to solve a general eigenvalue problem. This results in three different cases of possible solutions for the general eigenvalue problem. Using these cases, the eigenvalues and eigenfunctions can be calculated. The authors also present some examples. Two examples are verified in this chapter. The cantilever and free-free beam. For both the eigenvalues are calculated and compared to the results of \cite{VV06}. Some examples of modal shapes are also graphed. The idea of a suspended beam is breifly discussed. Next the article \cite{SP06} is discussed. In this section, the authors obtain eigenvalues of a free-free Timoshenko beam by vibrating and measuring the natural frequencies of a real beam. This empirical data is then compared to the theoretical results of various methods, including the Finite Element Method. In this last section, a example of the application of the Timoshenko beam model is presented. This section is a study of a technical report. In this report, the authors present a model for a cantilever Timoshenko beam with a tip body and elastic interfaces. This section also expands the numerical results of the report. The variational form of the model is derived, and the Galerkin approximation follows. The Mixed Finite Element Method is applied to the model. Using the Finite Element Method, the solution of the model is numerically calculated. Various results are given for different values of the parameters. The numerical results and figures confirms that the model behaves as expected. 

\subsubsection{Finite element method}
This chapter is preperation for the final chapter in this dissertation. The finite element method is applied to the models of this dissertation. Specifically, it is applied to the Cantilever model problem of the three-dimensional and two-dimensional elastic bosies, as well as the cantilever plate model. The cantilever Timoshenko beam model is excluded since chapter 4 provides an exact method to calculate the eigenvalues and eigenfunctions. Each of the models are assumed to have a rectangular cross-section and is made of the same isotropic material. For each model, a reference configuration is defined. The Galerkin approximation can then be derived and using the Finite Element Method, a system of oridinary differential equations can be obatined. At this point is it easy to derive a eigenvalue problem for each of the models. This eigenvalue problem can be solved using numerical methods.

\subsubsection{Validity of cantilever beam and plate models}

In this chapter, the validity of the models in this dissertation is investigated using numerical results. Specifically the validity of a cantilever Timoshenko beam and a cantilever Reissner-Mindlin plate is investigated. In this case, the term validity refers to how well the model compares to a more realistic model. All models are assumed to have a rectangular cross-section, are in a cantilever configuration and are made of the same isotropic material. The first section, is a discussion of the article \cite{LVV09}. In this article, the authors investigate the validity of the Timoshenko beam model. The authors compare the Timoshenko beam model to a more realistic two-dimensional beam model. The comparison is made with the eigenvalues and eigenfunctions. From modal analysis it is known that the solution of the models are linear combinations of the eigenvalues and eigenfunctions. The two-dimensional model also have eigenvalues that do not have a corresponding eigenvalue in the Timoshenko beam model. The authors call all matching eigenvalues beam-type eigenvalues. The authors show that the Timoshenko beam model compare well to the two-dimensional model for long and short beams. The results of the article are verified and extra results are also calculated for the rest of this chapter. In this section, the aricle \cite{LVV09} is extended to verify the validity of the two-dimensional elastic model. The two-dimensional elastic model is compared to a more realistic three-dimensional model. The two dimensional model serves as a intermediate step between the Timoshenko beam model and the three-dimensional model. The same approach of \cite{LVV09} is used to compare the two models. The result show that the two-dimensional model compare very well to the three-dimensional model, especially for beam problems. The last section takes the idea of \cite{LVV09} and extends it to the Reissner-Mindlin plate model. The Reissner-Mindlin plate model is compared to a more realistic three-dimensional plate model.\\



*******************************************************************************
%%%%%%%%%%%%%%%%%%%%%%%%%%%%%%%%%%%%%%%%%%%%%%%%%%%%%%%%%%%%%%%%%%%%%


\section{Comparison of models}
This chapter introduces the models used in this dissertation.

\subsection{Model problem for a three-dimensional elastic solid}

Two models for a three-dimensional elastic solid are presented. These models are a cantilever and a free-free elastic body. A dimensionless variational form of the cantilever model is derived. General plane stress is discussed.

\subsection{Two-dimensional model problem for an elastic solid}

A model for a two-dimensional elastic solid is presented. The model is a cantilever elastic body. This model is a special case of plane stress of a three-dimensional elastic body. The model is given in a dimensionless variational form by simplification of the three-dimensional model.

\subsection{Timoshenko beam models}

Four models for a one-dimensional beam are presented using the Timoshenko beam theory are presented. The models are a cantilever, free-free, pinned-pinned and a suspended beam. A dimensionless variational form for the models are derived.

\subsection{Reissner-Mindlin Plate Model}

Two models for a Reissner-Mindlin plate are presented. The models are a cantilever and a rigidly clamped plate. A dimensionless variational form for the cantilever model is derived.

\subsection{Rod Theory}
**************

\section{Mathematical analysis of vibration problems}

This chapter is a discussion and example of application of the theory presented in the article \cite{VV02}. This article concerns the existence and uniqueness of solutions of second order hyperbolic type problems. The models in this dissertation as such problems.\\

This chapter also concerns the idea of modal analysis. This is a method to find the natural frequencies of a system. This method is used later in the dissertation.

\subsection{Existence and uniqueness of solutions}
This section is a discussion of the theory presented in the article \cite{VV02}.

\subsubsection{The variational approach}

The authors define a general variational problem. This problem is defined on specific Hilbert spaces. Four assumptions are made in the article as well as some notation introduced.

\subsubsection{Main results for existence and uniqueness}
The main results of the article are given in this section. The results include existence and uniqueness of solutions for strong and weak damped models. The models in this dissertation are all weakly damped. 

\subsubsection{First order system}
To prove the results, the approach of the authors are as follows.\\

The general variational problem is rewritten for a first order system of differential equations. This system is defined on a Hilbert space $H$. A operator $A$ is defined, and using the assumptions the authors prove that the operator $A$ is an infinitesimal generator of a $C_0$-semigroup of contractions with a domain in $H$.\\

Using this operator the authors rewrite the system of differential into the form $\displaystyle x' = Ax + f$. To be more precise, the system of first order system of differential equations can be written as an initial value problem.\\

The existence and uniqueness results for the first order system is easily proven with an initial value problem. The authors also prove that a solution of the first order system is also a solution to the general variational problem.\\

Using semi-group theory, a unnique solution for the initial value problem can be proved. The authors also provide the link between the solutions of the initial value problem and the general variational form. Thus proving the required results.

\subsection{Application: Timoshenko beam model}
As an example of the application of the theory, the weak variational form for the cantilever Timshenko beam is introduced. This weak variational problem is defined on complete product spaces.\\

The four assumptions of \cite{LVV09} are proved for this example model.

\subsection{Modal analysis}
*****************************************************************

\subsubsection{Timoshenko beam}
\subsubsection{General vibration problem}
\section{Finite element theory}

\subsection{Convergence of the Galerkin approximation}
In this Chapter, two convergence results related to the Finite Element Method is presented. This first result is the convergence of the Galerkin approximation of a second order hyperbolic type problem, presented in the article \cite{BV13}.\\

The second result concerns the convergence of the eigenvalue and eigenfunctions of a one-dimensional second order hyperbolic type problem when applying the Finite Element Method. This is presented in the textbook \cite{SF97}.

\subsubsection{Galerkin approximation}
In this section, the authors use the same general variational problem as presented in \cite{VV02} (Chapter 2). The authors assume that this general variational problem has a unique solution.\\

To start, the Galerkin approximation for the general variational problem is derived. The a unique solution is proved in \cite{BV13} when the forcing function $f$ is continuous.\\

\subsubsection{System of ordinary differential equations}

Using the Finite Element Method matrices, the Galerkin approximation is rewritten as a system of ordinary differential equations.

\subsubsection{Error estimates}
Given that $u$ is a solution of the general variational problem, and $u^h$ is a solution to the Galerkin approximation, the authors define the error $e^h = u - u^h$.\\

The in \cite{BV13} the authors prove that this error can be made arbitrarily small by choosing a sufficiently small $h$, such that $u^h \rightarrow u$ as $h \rightarrow 0$. \cite{BV18} extends \cite{BV13} to include damping.\\


To obtain the error estimate, the authors split the error into two parts. The first part is a semi-dicrete problem and the second is a discrete problem. These two problems are investigated seperately.

\subsubsection{Main result}

In this section, the main results of \cite{BV13} are presented. The two assumptions made by the authors are also given.

\subsubsection{Obtaining the error estimate}

In this section, the approach of the authors for determining the error estimates for the semi-discrete and fully discrete problems are presented.\\


\subsection{Convergence of eigenvalues in Finite Element Analysis}
*****************************************************************

\subsubsection{Eigenvalue problem for vibration problem}
\textbf{Problem GVarE}
\subsubsection{Interpolation theorem}
\subsubsection{Eigenvalue problem for our models*********}
\textbf{Notation}
\subsubsection{Convergence of the eigenvalues}
\subsubsection{Convergence of the eigenfunctions}
\subsection{Example}
\textbf{Problem 2D-1V}\\
\textbf{Test functions}
\subsubsection{Weak variational form}
\textbf{Problem 2D-1W}
\section{Timoshenko beam model}

In this chapter, the Timoshenko beam theory is explored further. In the first three sections, the theory and two examples of \cite{VV06} are given. 
\subsection{Eigenvalue problem}

In this article \cite{VV06}, the authors present a method for finding the exact eigenvalues and eigenfunctions for the Timoshenko beam theory. The method provides solutions for three cases where $\lambda <\alpha$, $\lambda =\alpha$ and $\lambda >\alpha$.\\

\subsection{Cantilever beam}
This is an example of the application of the theory on a cantilever Timoshenko beam. In this example some results in \cite{VV06} is verified and examples of the mode shapes are given.

\subsection{Free-free timoshenko beam}

This is an example of the application of the theory on a free-free Timoshenko beam.In this example some results in \cite{VV06} is verified and examples of the mode shapes are given.

\subsection{Suspended beam model}
The idea of a suspended Timoshenko is also discuss briefly.

\subsection{SP06}
In this section this section, the article \cite{SP06} is discussed. In this section, the authors obtain eigenvalues of a free-free Timoshenko beam by vibrating and measuring the natural frequencies of a real beam. This empirical data is then compared to the theoretical results of various methods, including the Finite Element Method.

\subsection{Cantilever Timoshenko beam with tip-body}

In this last section, a example of the application of the Timoshenko beam model is presented. This section is a study of a technical report. In this report, the authors present a model for a cantilever Timoshenko beam with a tip body and elastic interfaces. This section also expands the numerical results of the report.\\

The variational form of the model is derived, and the Galerkin approximation follows. The Mixed Finite Element Method is applied to the model.\\

Using the Finite Element Method, the solution of the model is numerically calculated. Various results are given for different values of the parameters. The numerical results and figures confirms that the model behaves as expected.

\section{Finite element method}
This chapter is preperation for the final chapter in this dissertation. The finite element method to a cantilever version of each of the models in this dissertation. The goal of the chapter is to get a numerical formula to solve the eigenvalue problem for each model. The Timoshenko beam model is excluded from this chapter since an exact methd is already presented in chapter 4 (\cite{VV02}).\\

\subsection{Two-dimensional cantilever body}

The model is assumed to have a rectangular cross-section and is in a cantilever configuration. The Galerkin approximation is derived and applying the finite element method, a numerical formula for solving the model problem is given.\\

In this numerical formula, the model problem is a system of linear equations. This system can be changed to obtain a numerical formula for teh eigenvalue problem.\\


\subsection{Three-dimensional elastic model}
Similar to above.

\subsection{Cantilever plate model}
Similar to above.

\section{Validity of cantilever beam and plate models}

In this chapter, the validity of the models in this dissertation is investigated using numerical results. Specifically the validity of a cantilever Timoshenko beam and a cantilever Reissner-Mindlin plate is investigated.\\

In this case, the term validity refers to how well the model compares to a more realistic model.\\

All models are assumed to have a rectangular cross-section, are in a cantilever configuration and are made of the same isotropic material.\\




\subsection{Validity of the Timoshenko beam model}
This first section, is a discussion of the article \cite{LVV09}. In this article, the authors investigate the validity of the Timoshenko beam model. The authors compare the Timoshenko beam model to a more realistic two-dimensional beam model.\\

The comparison is made with the eigenvalues and eigenfunctions. From modal analysis it is known that the solution of the models are linear combinations of the eigenvalues and eigenfunctions.\\

The two-dimensional model also have eigenvalues that do not have a corresponding eigenvalue in the Timoshenko beam model. The authors call all matching eigenvalues beam-type eigenvalues.\\

The authors show that the Timoshenko beam model compare well to the two-dimensional model for long and short beams. The results of the article are verified and extra results are also calculated for the rest of this chapter.\\

\subsection{Validity of the two-dimensional elastic model}

In this section, the aricle \cite{LVV09} is extended to verify the validity of the two-dimensional elastic model. The two-dimensional elastic model is compared to a more realistic three-dimensional model.\\

The two dimensional model serves as a intermediate step between the Timoshenko beam model and the three-dimensional model.\\

The same approach of \cite{LVV09} is used to compare the two models. The result show that the two-dimensional model compare very well to the three-dimensional model, especially for beam problems.


\subsection{Validity of the Reissner-Mindlin plate model}

The last section takes the idea of \cite{LVV09} and extends it to the Reissner-Mindlin plate model. The Reissner-Mindlin plate model is compared to a more realistic three-dimensional plate model.\\



******************************************************************\\

\section{Comparison of models}
\subsection{Model problem for a three-dimensional elastic solid}
\subsubsection{Equations of motion and constitutive equations}
\textbf{Equation of motion}
\subsubsection{Dimensionless form}
\subsubsection{Model problems}
\textbf{Problem 3D-1}\\
\textbf{Problem 3D-2}
\subsubsection{Variational form}
\textbf{Bilinear forms and integral}\\
\textbf{Test functions}\\
\textbf{Problem 3D-1V}
\subsubsection{Plane stress}
\subsection{Two-dimensional model problem for an elastic solid}
\subsubsection{Equations of motion and constitutive equations}
\textbf{Equation of motion}\\
\textbf{Constitutive equation}
\subsubsection{Model problem}
\textbf{Problem 2D-1}
\subsubsection{Variational form}
\textbf{Bilinear forms and integral}\\
\textbf{Test functions for problem 2D-1}\\
\textbf{Problem 2D-1V}
\subsection{Timoshenko beam models}
\subsubsection{Equations of motion and constitutive equations}
\subsubsection{Boundary conditions}
\subsubsection{Boundary value problems}
\subsubsection{Variational form}
\textbf{Function spaces}\\
\textbf{Test function spaces for different problems}\\
\textbf{Bilinear forms}
\subsection{Reissner-Mindlin Plate Model}
\subsubsection{Equations of Motion and Constitutive Equations}
\subsubsection{Dimensionless Form}
\textbf{Remark}\\
\textbf{Dimensionless Equations of Motion}\\
\textbf{Dimensionless Constitutive Equations}
\subsubsection{Boundary Conditions}
\subsubsection{Model Problems}
\textbf{Problem P-1}\\
\textbf{Problem P-2}
\subsubsection{Variational Form}
\textbf{Bilinear Forms and integral}\\
\textbf{Test Functions for Problem P-1}\\
\textbf{Problem P-1V}
\subsection{Rod Theory}
\section{Mathematical analysis of vibration problems}
\textbf{Problem T-2W}
\subsection{Existence and uniqueness of solutions}
\subsubsection{The variational approach}
\subsubsection{Main results for existence and uniqueness}
\subsubsection{First order system}
\subsection{Application: Timoshenko beam model}
\subsection{Modal analysis}
\subsubsection{Timoshenko beam}
\subsubsection{General vibration problem}
\section{Finite element theory}
\subsection{Convergence of the Galerkin approximation}
\subsubsection{Galerkin approximation}
\subsubsection{System of ordinary differential equations}
\subsubsection{Error estimates}
\subsubsection{Main result}
\subsubsection{Obtaining the error estimate}
\textbf{Semi-discrete problem}
\subsubsection{Fully discrete Galerkin finite element approximation}
\textbf{Notation}\\
\textbf{Error estimate for the fully discrete problem}
\subsection{Convergence of eigenvalues in Finite Element Analysis}
\subsubsection{Eigenvalue problem for vibration problem}
\textbf{Problem GVarE}
\subsubsection{Interpolation theorem}
\subsubsection{Eigenvalue problem for our models*********}
\textbf{Notation}
\subsubsection{Convergence of the eigenvalues}
\subsubsection{Convergence of the eigenfunctions}
\subsection{Example}
\textbf{Problem 2D-1V}\\
\textbf{Test functions}
\subsubsection{Weak variational form}
\textbf{Problem 2D-1W}
\section{Timoshenko beam model}
\subsection{Eigenvalue problem}
\textbf{General eigenvalue problem}\\
\textbf{Case $\bm {\lambda <\alpha }$}\\
\textbf{Case $\bm { \lambda =\alpha }$}\\
\textbf{Case $\bm {\lambda >\alpha }$}
\subsection{Cantilever beam}
\subsubsection{Calculating the eigenvalues}
\subsubsection{Example of mode shapes}
\subsection{Free-free timoshenko beam}
\subsubsection{Calculating the eigenvalues}
\subsubsection{Example of mode shapes}
\subsection{Suspended beam model}
\textbf{Variational problem}
\subsection{SP06}
\subsubsection{Mathematical models}
\textbf{Problem 3D-2}
\subsubsection{Experimental setup}
\subsubsection{Results from SP06}
\subsection{Cantilever Timoshenko beam with tip-body}
\subsubsection{The model problem}
\textbf{Boundary conditions}\\
\textbf{Remark}
\subsubsection{Variational form}
\textbf{Remark}\\
\textbf{Interface equation}\\
\textbf{Problem Var}
\subsubsection{Galerkin approximation}
\textbf{Problem G}
\subsubsection{System of ordinary differential equations}
\textbf{Problem ODE}
\subsubsection{Numerical results}
\textbf{Case: Control}\\
\textbf{Case: Decrease $\gamma $}\\
\textbf{Case: Decrease $\mu $}\\
\textbf{Case: Increase $m$}
\section{Finite element method}
\subsection{Two-dimensional cantilever body}
\textbf{Reference configuration for rectangular cross-section}\\
\textbf{Cantilever elastic body}\\
\textbf{Problem 2D-1V}\\
\textbf{Constitutive equations:}\\
\textbf{Bilinear form:}
\subsubsection{Weak variational form}
\textbf{Problem 2D-1WV}
\subsubsection{Galerkin approximation}
\textbf{Problem 2D-1G}
\subsubsection{System of differential equations}
\textbf{FEM matrices}\\
\textbf{Problem 2D-1ODE}\\
\textbf{Remark}
\subsubsection{Eigenvalue problem}
\textbf{Problem 2D-1E}
\subsection{Three-dimensional elastic model}
\textbf{Reference configuration for rectangular cross-section}\\
\textbf{Cantilever elastic body}\\
\textbf{Problem 3D-1V}\\
\textbf{Bilinear Form}
\subsubsection{Weak variational form}
\textbf{Bilinear form}\\
\textbf{Problem 3D-1W}
\subsubsection{Galerkin approximation}
\textbf{Problem 3D-1G}
\subsubsection{System of ordinary differential equations}
\textbf{FEM matrices}\\
\textbf{Problem 3D-1ODE}
\subsubsection{Eigenvalue problem}
\textbf{Problem 3D-1E}
\subsection{Cantilever plate model}
\textbf{Reference configuration for a rectangular plate}\\
\textbf{Cantilever plate model}\\
\textbf{Problem P-1V}\\
\textbf{Constitutive Equations}\\
\textbf{Bilinear Form}
\subsubsection{Weak variational form}
\textbf{Bilinear forms}\\
\textbf{Problem Plate-1W}\\
\textbf{Existence and uniqueness}
\subsubsection{Galerkin approximation}
\textbf{Problem P-1G}
\subsubsection{System of ordinary differential equations}
\textbf{FEM matrices}\\
\textbf{Problem P-1ODE}
\subsubsection{Eigenvalue problem}
\textbf{Problem P-1E}
\section{Validity of cantilever beam and plate models}
\textbf{Parameters}
\subsection{Validity of the Timoshenko beam model}
\subsubsection{The models}
\subsubsection{Accuracy of the numerical eigenvalues}
\subsubsection{Comparing the shape of the eigenfunctions}
\textbf{Shapes relating to beam-type eigenvalues}\\
\textbf{Shapes relating to non-beam type eigenvalues}\\
\textbf{Shapes of the cross-section}\\
\textbf{Direct comparison of mode shapes}\\
\textbf{Remark}
\subsubsection{Comparing the eigenvalues}
\textbf{Results of [LVV09]}
\subsubsection{Comparison of the eigenvalues}
\subsection{Validity of the two-dimensional elastic model}
\subsubsection{The models}
\subsubsection{Accuracy of the numerical eigenvalues}
\subsubsection{Comparing the mode shapes}
\textbf{Mode shapes relating to beam type eigenvalues.}\\
\textbf{Mode shapes relating to non-beam type eigenvalues that are present in the two-dimensional model.}\\
\textbf{Mode shapes relating to non-beam type eigenvalues that are not present in the two-dimensional model.}
\subsubsection{Comparing the eigenvalues}
\subsubsection{$b > h$}
\subsection{Validity of the Reissner-Mindlin plate model}
\subsubsection{Comparing the mode shapes}
\subsubsection{Comparing the eigenvalues}
\section{Overview}
\subsection{Overview}
\section{Conclusion}
\subsection{Overview}
\textbf{Properties}\\
\textbf{Properties of Type One}\\
\textbf{Properties of Type Two}\\
\textbf{Properties}
\textbf{Different Research Approaches}\\
\textbf{Motivation}
\subsection{NRF Research Proposal}
\subsection{Modelling a Suspension Bridge: A Brief History}
\subsection{Modelling a suspension bridge: An Engineering Example}
\subsection{Simplified Mathematical Model for a Suspended Beam}
\subsubsection{Reference Configuration}
\subsubsection{System of Equations}
\textbf{Equations of Motion}\\
\textbf{Constitutive Equations}\\
\textbf{Dimensionless Equations}
\subsubsection{Boundary Conditions}
\subsection{Analysis of a Thermoelastic Timoshenko Beam Model}
\textbf{Dimensionless Equations}
\subsection{An Elastically Clamped Beam Model}
\textbf{Boundary Conditions}\\
\textbf{Hooke's law in two-dimensions}








\begin{itemize}
	\item Chapter 1: Models, Dimensionless, Variational Forms, Modal Analysis
	\item Chapter 2: Example Weak Variational, Existence and Uniqueness from [VV02], Application to example
	\item Chapter 3: Example Galerkin, Convergence of Galerkin [BV13], Application to example, Convergence of Eigenvalues using FEM
	\item Chapter 4: Eigenvalues of Timo [VV06], Example of applications on models, Emperical Study of validity [SP06], Cantilever with Tip-Body (Example why need validity)
	\item Chapter 5: FEM of 2D, 3D, and Plate
	\item Chapter 6: Validity of 1D, Validity of 2D, Validity of Plate
	\item Chapter 7: Conclusion
\end{itemize}


In chapter 1, word al die modelle gegee wat in die dissertation gebruik word. Die modelle word in dimenielose form herskryf. Die modelle is dan herskryd in variational form wat gebruik gaan word vir die res van die verhandeling. Modal analysis word ook in chapter 1 verduidelik in hand vn 'n voorbeeld.\\

 In Chapter 2 word die werk van die artikel [VV02] gebruik om na die existence and uniqueness of solutions van second order hyperbolic equations te kyk. Die cantilever beam word as 'n voorbeeld gebruik om die werk te verduidelik.\\
 
 Chapter 3 kyk na die convergence van die FEM. Dit volg die werk van die artikel [XX]. Die methode van die artikel is om error estimates vir die fully discrete vartiational problem (Galerkin Approximation) te bepaal. In die tweede deel van chapter 3, word daar na die convergence van die eigenvalues en eigenvectors in die FEM gekyk. Dit volg die werk van die handboek [XX] en word in beter notation oorgeskryf.\\
 
 Chapter 4 fokus op the Timoshenko beam. Eerste word daar gekyk na die eigenvalue problem van die Timoshenko beam theory. Die artikel [VV06] gee die nodige werk om die eigenvalues en eigenvectors exactly uit te werk. Die methode word dan op 'n free-free beam en ook 'n cantilever beam toe gepas waar die eiewaarde uitgewerk word en ook die eigenfunctions geskets word. Daar is dan navorsing gedoen deur [SP06] wat gekyk het hoe goed 'n Timoshenko beam theory vergelyk met 'n 3D en 'n regte balk. Die navorsing wys dat die Timoshenko goed vergelyk die die regte balk in 'n free-free configuration. Die navorsing van [SP06] word discuss en gebruik dit as motiveering vir verdere numeriese simulation. Laaste deel in chapter 4 is 'n voorbeeld van 'n Timoshenko beam in 'n model wat meer real world application het. Dit is in die tegniese report van [XX]. Die werk in die chapter oor die tip body balk is net gefokus op die numeriese simulation.\\
 
 Chapter 5 gaan oor die plaat model. Die chapter doen die FEM analiese van die cantilever model. Die chapter le oook die reference configuration uit vir 'n rectangular plate model en adress die shear correction factor. Kan moontlik deel word van die volgende chapter.\\
 
 Chapter 6 is dieselfde as die vorige chapter Die chapter hanteer die FEM analise vir die twee en drie dimensionele elastic bodies. Die reference configuration is vir 'n rectangular form vir beide sodat dit later gebruik kan word indie volgende chapter. Die chapter gee ook die eigenvalue problems en wys wys dat die regte accuracy kan obtained word die die eiewaardes numeries uit te werk.\\
 
 Die laaste chapter hanteer die hoof numeriese werk van die verhandeling. Die die chapter word die validityvna die timoshenko beam asook die plaat model ondersoek. Eerste word die timoshenko beam met 'n twee-D beam vergelyk. Die is dieselfde as die artikel [XX]. So dieselfde werk word gedoen as die artikel, maar met aner waardes. Die werk word dan extend deur die twee-D model te vergelyk met die 3D model. Die modelle is elastic bodies, so daar work ook apart na die beam-type en non-beam type dele gekyk. Die eiewaardes word vergelyk met verskillende diktes va die 3D model en dan word daar gekyk na die eiefunksies se forms. Dan 
\end{document}
