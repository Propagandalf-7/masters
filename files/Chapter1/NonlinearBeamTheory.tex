\documentclass[../../main.tex]{subfiles}
\setcounter{chapter}{1}
\setcounter{section}{6}
\begin{document}
\section{Rod Theory}
In the literature the word rod is often used for a one-dimensional
continuum for example linear and nonlinear beams but also hoses
..... ................ .............. Cosserat .........
[LA12]

J C Simo and L Vu-Quoc published a number of articles on beams and
plates. What is interesting to us here is [SV87] where they consider
``The role of nonlinear theories in transient dynamic analysis of
flexible structures". They state that large motion of beams and
plates has been studied since 1958 and consider 12 publications up
to 1986. The authors discuss shortcomings and wrong results in some
detail. What we also find interesting is that the article [Ant76] is
not mentioned.

It is clear from [Ant72] ~~ +  ~ [Ant76] that Antman follows a
rigorous mathematical approach to the theories on beams and plates.
Antman published a {\bf survey} article [Ant96] on nonlinearly
viscoelastic and elastic rods. According to him, some equations of
rods attained a form suitable for mathematical analysis in the ten
years preceding his 1996 paper. The articles he referred to were all
published by J C Simo and L Vu-Quoc. In our opinion this means that
he suggests a study of the connections between the two approaches
worth while.

A study of all the results mentioned above to ..................
.........\\ ................................................ \\
is beyond the scope of this literature study (especially since we
also aim to include detailed applications). ...................
\\ the aim of this study is comparison of different models for the
same application. A start is to restrict the study to planar motion
of a beam where the material is linearly elastic as in [VDB16].
\end{document}
