\documentclass[../../main.tex]{subfiles}
\begin{document}
\section{Model problem for a three-dimensional elastic solid} \label{sec:3D_Model}
In this section we introduces the linear theory for a three-dimensional elastic
solid. The textbook of Fung \cite{Fung65} is used with some changes in the
notation. \\

The equation of motion and constitutive equation are given in subsection
\ref{ssec:3D_Model:EquationOfMotion}. The dimensionless form of the equation of
motion and constitutive equation are derived in subsection
\ref{ssec:3D_Model:DimensionlessForm}. Model problems are presented in
subsection \ref{ssec:3D_Model:ModelProblems}. The variational form for the
vibration problem is given in section \ref{ssec:3D_Model:VariationalForm}.
Plane stress is discussed in section \ref{ssec:3D_Model:PlaneStress}

\subsection{Equations of motion and constitutive equations} \label{ssec:3D_Model:EquationofMotion+ConstitutiveEquation}
Consider a vector valued function $u$ defined on domain $\Omega \subset {R}^3$
where $u$ describes the displacement of $\Omega$ in ${R}^3$.
\subsubsection{Equation of motion}\label{ssec:3D_Model:EquationOfMotion}
\begin{eqnarray}
	\rho\partial_t^2 u & = & \textrm{div}T + Q. \label{eq:3D_Model:EM}
\end{eqnarray}

In \eqref{eq:3D_Model:EM}, $\rho$ is the density of the elastic body, $T$ is
the stress tensor with components $\sigma_{ij}$, $Q$ is an external body force
acting on $\Omega$ and $\textrm{div}T$ is the divergence of the tensor $T$ and
represented by
\begin{eqnarray}
	\textrm{div}  T & = &
	\begin{bmatrix}
		\partial_1 \sigma_{11} + \partial_2 \sigma_{12} + \partial_3 \sigma_{13} \\
		\partial_1 \sigma_{21} + \partial_2 \sigma_{22} + \partial_3 \sigma_{23} \\
		\partial_1 \sigma_{31} + \partial_2 \sigma_{32} + \partial_3 \sigma_{33}
	\end{bmatrix}. \label{eq:3D_Model:divT}
\end{eqnarray}

Only small vibrations are considered. This means that the local displacements
and rotations are small. Hence the stress tensor $T$ is symmetric. Let
$\textrm{Tr}(T)$ denote the trace of the stress tensor $T$, that is
\begin{eqnarray}
	\textrm{Tr}(T) & = & \phi_{11} + \phi_{12} + \phi_{13}. \label{eq:stress_tensor_t}
\end{eqnarray}

\textbf{Remark} Some books prefer to use $\rho Q$ for the external body force, but $Q$ is also correct. $Q$ is then a force per unit volume.

\subsubsection*{Strain}\label{sssec:3D_Model:Strain}
Infinitesimal strain tensor is defined on $\Omega$ as $\mathcal{E}$, with components
\begin{eqnarray}
	\varepsilon_{ij} = \frac{1}{2}\left( \frac{\partial u_i}{\partial x_j} + \frac{\partial u_j}{\partial x_i} \right). \label{eq:3D_Model:Strain}
\end{eqnarray}

In this dissertation, we only consider isotropic materials. As a consequence,
the constitutive equation (Hooke's law) takes the following form.

\subsubsection*{Hooke's law in terms of $E$ and $\nu$}\label{sssec:3D_Model:Hooke'sLaw1}
\begin{eqnarray}
	\mathcal{E} = \left( \frac{1+\nu}{E} \right)T - \frac{\nu}{E}\textrm{Tr}(T)I,  \label{eq:3D_Model:HL}
\end{eqnarray} where $E$ is Young's Modulus and $\nu$ is Poisson's ratio.

\subsubsection*{Hooke's law in the alternative form}\label{sssec:3D_Model:Hooke'sLaw2}
If the principal stresses $\sigma_i$ are all non-zero, then Hooke's law can be written in the following form
\begin{eqnarray}
	T = \left( \frac{E}{1+\nu} \right)\mathcal{E} + \frac{\nu E}{(1+\nu)(1-2\nu)}\textrm{Tr}(\mathcal{E})I \label{eq:3D_Model:CE}.
\end{eqnarray}
$\textrm{Tr}(\mathcal{E}) = \epsilon_{11} + \epsilon_{12} + \epsilon_{13}$ is the trace of the strain operator $\mathcal{E}$.\\

Hooke's Law \eqref{eq:3D_Model:CE} is the constitutive equation for the
three-dimensional elastic model used in problems. Conditions for the problems
to be well posed, are discussed in Chapter 2.

\subsection{Dimensionless form} \label{ssec:3D_Model:DimensionlessForm}
The dimensionless form of the equation of motion and constitutive equation are
derived in this subsection. Suppose $\ell$ represents some notable dimension
(e.g.the length) of the elastic body and $G$ the shear modulus of elasticity.\\

Set \[\tau = \frac{t}{t_0},\quad \xi_i = \frac{x_i}{\ell},\quad {u}^*(\xi,\tau) = \frac{1}{\ell}{u}(x,t), \quad Q^{*} = \ell G \kappa^2 Q, \]
\[ \text{and} \ \,\,\, \sigma_{ij}^*(\xi,\tau) = \frac{1}{G\kappa^2}\sigma_{ij}(x,t),\]

where $\kappa^2$ is a dimensionless constant and $t_0$ must be specified. A
convenient choice for $t_0$ (see section \ref{sec:1D_Model}) is \[t_0 = \ell\sqrt{\frac{\rho}{G\kappa^2}}.\]

Substitution into \eqref{eq:3D_Model:EM} and yields
\begin{eqnarray*}
	\partial_{\tau} u^{*} = \textrm{div}T^* + Q^*.
\end{eqnarray*}
where $T^* = \sigma_{i,j}^*(\xi,\tau)$.\\

The dimensionless form of \eqref{eq:3D_Model:HL} and \eqref{eq:3D_Model:CE} are
\begin{eqnarray*}
	\mathcal{E} = \frac{G \kappa^2}{E} \left[ (1+\nu)T^* - \nu \textrm{Tr}(T^*)I \right]
\end{eqnarray*}

and
\begin{eqnarray*}
	T^* = \frac{E}{G \kappa^2} \left[\left( \frac{1}{1+\nu} \right)\mathcal{E} + \frac{\nu}{(1+\nu)(1-2\nu)}\textrm{Tr}(\mathcal{E})I \right]
\end{eqnarray*}

Introduce a dimensionless constant
\begin{eqnarray*}
	\gamma= \frac{G\kappa^2}{E}.
\end{eqnarray*} Using this dimensionless constant,
\begin{eqnarray}
	\mathcal{E} = \gamma(1+\nu)T^* - \gamma\nu \textrm{Tr}(T^*)I \label{DM_H_E}
\end{eqnarray}
and
\begin{eqnarray}
	T^* = \frac{1}{\gamma(1+\nu)}\mathcal{E} + \frac{\nu}{\gamma(1+\nu)(1-2\nu)}\textrm{Tr}(\mathcal{E})I. \label{DM_H_T}
\end{eqnarray}

\textbf{Remark} The constant $G\kappa^2$ is introduced to allow for comparisons of the models in later chapters. The constant comes from the Timoshenko beam theory and is explained in section \ref{sec:1D_Model} \\

In the rest of section 1.2 the original notation is retained for convenience.

\subsubsection*{Equations of motion in dimensionless form}\label{sssec:3D_Model:EquationOfMotionDimensionless}
\begin{eqnarray}
	\partial_t^2 u & = & \textrm{div}T + Q \label{eq:3D_Model:EM-D}
\end{eqnarray}
with
\begin{eqnarray}
	\textrm{div}  T & = &
	\begin{bmatrix}
		\partial_1 \sigma_{11} + \partial_2 \sigma_{12} + \partial_3 \sigma_{13} \\
		\partial_1 \sigma_{21} + \partial_2 \sigma_{22} + \partial_3 \sigma_{23} \\
		\partial_1 \sigma_{31} + \partial_2 \sigma_{32} + \partial_3 \sigma_{33}
	\end{bmatrix}.\label{eq:3D_Model:divT-D}
\end{eqnarray}

\subsubsection*{Constitutive equations in dimensionless form}\label{sssec:3D_Model:ConstitutiveEquationsDimensionless}
\begin{eqnarray}
	T = \frac{1}{\gamma(1+\nu)} \mathcal{E} + \frac{\nu}{\gamma(1+\nu)(1-2\nu)}\textrm{Tr}(\mathcal{E})I \label{eq:3D_Model:CE-D}
\end{eqnarray}

\subsection{Model problems}\label{ssec:3D_Model:ModelProblems}
Suppose $\Omega \subset R_3$ is the reference configuration for a solid
executing small vibrations. The boundary of $\Omega$ can be divided into two
distinct parts, referred to as $\Sigma$ and $\Gamma$. The following will be
considered a model problem for a three-dimensional elastic body.
\begin{enumerate}
	\item[] The equations of motion \eqref{eq:3D_Model:EM-D} and \eqref{eq:3D_Model:divT-D} is satisfied in $\Omega$;
	\item[] Hooke's law \eqref{eq:3D_Model:CE-D} is satisfied in $\Omega$;
	\item[] The displacement $u = u_\Sigma$ is specified on
		$\Sigma$;
	\item[] The traction $Tn = t_\Gamma$ is specified on
		$\Gamma$.
\end{enumerate}

The two model problems considered in this dissertation are described below.

\subsubsection{Problem 3D-1}\label{sssec:3D_Model:Problem3D1}
Find a vector valued function $u$, satisfying equations
\eqref{eq:3D_Model:EM-D} to \eqref{eq:3D_Model:CE-D} and the following boundary
conditions:\\

Boundary Conditions:
\begin{eqnarray*}
	u & = & 0 \quad \textrm{ on } \Sigma\\
	Tn & = & 0 \quad \textrm{ on } \Gamma
\end{eqnarray*} With $n$ the outward normal vector to $\Omega$.

\subsubsection{Problem 3D-2}\label{sssec:3D_Model:Problem3D2}
Find a vector valued function $u$, satisfying equations
\eqref{eq:3D_Model:EM-D} to \eqref{eq:3D_Model:CE-D} and the following boundary
conditions:\\

Boundary Conditions:
\begin{eqnarray*}
	Tn & = & 0 \quad \textrm{ on } \partial\Omega
\end{eqnarray*} With $n$ the unit outward normal vector.

\subsection{Variational form}\label{ssec:3D_Model:VariationalForm}
Let $\phi \in C({\Omega})$ be an arbitrary vector valued function. Multiplying
$\phi$ to equation \eqref{eq:3D_Model:EM-D} and integrating over the domain
$\Omega$ results in,

\begin{eqnarray*}
	\int_{\Omega} (\partial_t^2 u)\cdot \phi \ dV & = & \int_{\Omega} (\textrm{div} T)\cdot \phi \ dV
	+ \int_{\Omega} Q \cdot \phi \ dV.
\end{eqnarray*}

Since the stress tensor T is symmetric,
\begin{eqnarray*}
	\textrm{div}(T\phi) & = & (\textrm{div}T)\cdot \phi + \textrm{Tr}(T\Phi),
\end{eqnarray*}
where
\begin{eqnarray*}
	\Phi & = &
	\begin{bmatrix}
		\partial_1 \phi_1 & \partial_2 \phi_1 & \partial_3 \phi_1 \\
		\partial_1 \phi_2 & \partial_2 \phi_2 & \partial_3 \phi_2 \\
		\partial_1 \phi_3 & \partial_2 \phi_3 & \partial_3 \phi_3
	\end{bmatrix},
\end{eqnarray*}
and
\begin{eqnarray*}
	\textrm{Tr}(T\Phi) & = & \sigma_{11} \partial_1\phi_1 + \sigma_{12}\partial_1 \phi_2 + \sigma_{13}\partial_1 \phi_3 + \sigma_{21}\partial_2 \phi_1 + \sigma_{22}\partial_2\phi_2 + \sigma_{23}\partial_2 \phi_3 \nonumber \\ & & + \sigma_{31}\partial_3 \phi_1 + \sigma_{32}\partial_3\phi_2 + \sigma_{33}\partial_3 \phi_3. \label{Trc}
\end{eqnarray*}

Using the Divergence Theorem and the symmetry of T,
\begin{eqnarray*}
	\int_{\Omega} \textrm{div}(T\phi) \ dV & = & \int_{\partial \Omega} T\phi \cdot n \ dS,\\ & = & \int_{\partial \Omega} Tn \cdot \phi \ dS.
\end{eqnarray*}

The divergence formula gives
\begin{eqnarray*}
	\int_{\Omega} \textrm{div}(T)\cdot \phi \ dV & = & -\int_{\Omega} \textrm{Tr}(T\Phi) \ dV + \int_{\partial \Omega} Tn\cdot \phi dS.
\end{eqnarray*}
Therefore,
\begin{align*}
	\int_{\Omega} (\partial_t^2 u)\cdot \phi \ dV = -\int_{\Omega} \textrm{Tr}(T\Phi) \ dV +
	\int_{\Omega} Q\cdot\phi \ dV + \int_{\partial \Omega} Tn\cdot \phi \ dS.
\end{align*}
\\

The general variational form of the three-dimensional model is given as
\begin{eqnarray*}
	\int_\Omega~ (\partial^2_t u) \cdot \phi~dV & = & \int_\Omega~c_1\textrm{Tr}({\cal E}\Phi)+
	c_2\textrm{Tr}({\cal E}) \textrm{Tr}(\Phi) ~dV + \int_\Omega~ Q \cdot \phi~dV +
	\int_{\partial \Omega}~Tn \cdot \phi~dS,
\end{eqnarray*}
with $ \displaystyle c_1 = \frac{1}{\gamma(1+\nu)}$ and $\displaystyle c_2 = \frac{\nu}{\gamma(1+\nu)(1-2\nu)}$.\\

\subsubsection{Bilinear forms and integral}\label{sssec:3D_Model:BilinearForm}
Define the bilinear forms
\begin{eqnarray}
	b(u,\phi) = \int_\Omega~c_1\textrm{Tr}({\cal E}\Phi)+ c_2\textrm{Tr}({\cal E})
	\textrm{Tr}(\Phi) ~dV \label{eq:3D_Model:Bilinear}
\end{eqnarray}
and
\begin{eqnarray}
	c(u,\phi) = \int_\Omega~ (\partial^2_t u) \cdot \phi~dV \label{eq:3D_Model:Bilinear_c}
\end{eqnarray}
with $\displaystyle c_1 = \frac{1}{\gamma(1+\nu)}$ and $\displaystyle c_2 = \frac{\nu}{\gamma(1+\nu)(1-2\nu)}$.\\

Also define the integral
\begin{eqnarray}
	(f,g) &=& \int_{\Omega} f\cdot g \ dV \label{eq:3D_Model:Bilinear_int}
\end{eqnarray}

\subsubsection{Test functions}\label{sssec:3D_Model:TestFunction}
Define the following set of test functions $T(\Omega)$ for Problem 3D-1 and
Problem 3D-2:\\

Problem 3D-1
\begin{eqnarray*}
	T(\Omega) & = & \left\{ \phi \in C^1(\bar{\Omega}) \ | \ \phi = 0 \ \textrm{ on } \ \Gamma \right\}
\end{eqnarray*}

Problem 3D-2
\begin{eqnarray*}
	T(\Omega) & = &  C^1(\bar{\Omega})
\end{eqnarray*}

The variational problem of Problem 3D-1 is given as Problem 3D-1V.

\subsubsection{Problem 3D-1V}\label{sssec:3D_Model:Problem3D1V}
Find a function $u$ such that for all $t>0$, $u \in T(\Omega)$ and
\begin{align}
	c(u,\phi) = -b(u,\phi) + (Q,\phi), \label{eq:3D_Model:Problem3D1VEq}
\end{align}
for all $\phi \in T(\Omega)$\\

The well-posedness of the model is treated in Chapter 2, where Korn's
inequality is presented. Applications are continued in Chapter 4, 5 and 6. In
Chapter 4, a free-free three-dimensional beam is discussed as part of an
empirical study. In Chapter 5, the finite element analysis is applied to a
cantilever beam to solve the eigenvalue problem. In Chapter 6, a cantilever
beam in used in the comparison of our linear models.

\subsection{Plane stress}\label{ssec:3D_Model:PlaneStress}
When the stresses in an elastic body act parallel to a single plane, it is
referred to as plane stress.\\

Consider the right-hand orthonormal set $\left\{e_1, e_2, e_3\right\}$. Without
loss of generality, assume the stresses act parallel to the $e_1$-$e_2$ plane
i.e. $\sigma_{3i} = 0$ for all $i$. From Hooke's Law \eqref{DM_H_E} the
following equations are obtained:

\begin{equation}
	\begin{aligned}
		\varepsilon_{11} & =  \gamma  ( \sigma_{11} - \nu \sigma_{22}) \qquad \qquad \varepsilon_{33} & = & - \gamma \nu (\sigma_{11} + \sigma_{22})          \\
		\varepsilon_{22} & =   \gamma (\sigma_{22} - \nu\sigma_{11}) \qquad \qquad \varepsilon_{12}   & = & \  \gamma (1+\nu) \sigma_{12} \label{strain_comp}
	\end{aligned}
\end{equation}

and a sufficient condition
\begin{eqnarray}
	\varepsilon_{13} =  \varepsilon_{23} = 0.
\end{eqnarray}

After some manipulation, we verify the following equations for the stress
components given in \cite{Fung65}.

\noindent
\begin{minipage}{.5\linewidth}
	\begin{eqnarray*}
		\sigma_{11} & = & \frac{1}{\gamma(1-\nu^2)} ( \varepsilon_{11} + \nu\varepsilon_{22})\\
		\sigma_{22} & = & \frac{1}{\gamma(1-\nu^2)} (\nu \varepsilon_{11} + \varepsilon_{22}).
	\end{eqnarray*}
\end{minipage}%
\begin{minipage}{.5\linewidth}
	\begin{eqnarray}
		\sigma_{12} & = & \frac{1}{\gamma(1+\nu)} \varepsilon_{12} \label{stress_comp}
	\end{eqnarray}
\end{minipage}\\

Another necessary strain condition for plane stress is found by substituting
$\sigma_{11}$ and $\sigma_{22}$ from \eqref{stress_comp} into
$\varepsilon_{33}$ in \eqref{strain_comp} to obtain
\begin{eqnarray}
	\varepsilon_{33} = -\frac{\nu}{1-\nu} (\varepsilon_{11} + \varepsilon_{22}).\label{plane_stress_neseccary_condition}
\end{eqnarray}

This is known as general plane stress. In this dissertation only a special case
of general plane stress is considered in section \ref{sec:2D_Model}.

\end{document}