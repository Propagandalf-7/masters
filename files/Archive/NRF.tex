\documentclass[../../main.tex]{subfiles}
\begin{document}
\setcounter{chapter}{1}
\setcounter{section}{1}
\section{NRF Research Proposal}
\subsection*{Finite Element Analysis of Model Problems for Elastic Rods and Shells}
\subsubsection*{Research rationale and motivation}
From small transverse vibration of a linearly elastic (flat) plate [Mindlin, 1951], to a fully non-linear shell model, numerous mathematical problems are possible. (The same remark is also true for beams.) In this research project, our focus is on the small vibrations but not sufficiently small to justify a linear model.\\

In the article of [Antman,1976], we see that the author takes a rigorous mathematical approach to obtain fully general theories of non-linearly elastic rods and shells from three-dimensional theories. In a follow up article [Antman,1996], open problems of nonlinear elastic rods are investigates and references are made to 4 articles of Simo and Vu-Quoc. Where [Antman,1976] follows a vigorous math approach, [Simo and Vu-Quoc,1987] follows an engineering approach. Their work however, received wide recognition and it is worthwhile to compare their articles to the work of [Antman,1976] and [Antman,1996].\\

\subsubsection*{Problem Identification}
In applications simplified models can be used with some additional but different assumptions and it leads to different models for the same real life problem. An example of where numerous models are used to study the same real world problem is the Tacoma Narrows Bridge disaster of 1940. [Gazzola, 2013] and [McKenna,1999] both derive models for this specific problem but with different approaches. Even though both authors agree on the shortcomings of many bridge models and that non-linear theories should be used, [Gazzola, 2013] insists on using a plate to model the bridge, while [McKenna,1999] uses a beam. This gives at least two different models for the same real world problem.\\

\subsubsection*{Research aims and objectives}
The main objective of this literature study is the comparison of such models, using either analysis or a finite element method simulation. Adding assumptions to a general model leads to many different simplified models that can be applied to the same situation. It is important to investigate the differences in these models and their applicability to real world problems.\\

Calculating solutions for these models will be useful for the comparison, but due to the complexities of the problems, numerical methods are used to obtain approximate solutions. The finite element method is generally accepted to be the best for elasto-mechanics. Appropriate articles will be used to ensure convergence of the solutions.\\

This study will also look at possible simplifications to the models through the addition of more assumptions. Examples of simplifications to models is the added assumption of rigidity of a beam model. This simplifies the model from a system of partial differential equations, to a system of ordinary differential equations.\\

This masters research is an in depth literature study with a main focus on model problems on shells and rods with attention given to:
\begin{itemize}
 \item Comparison of different models for the same problem either using analysis or numerical methods.
 \item Determining existence of solutions, given that the relevant articles are available.
 \item Convergence of finite element method when calculating solutions to the model problems.
\end{itemize}
\end{document}
