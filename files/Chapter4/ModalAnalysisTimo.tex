\documentclass[../../main.tex]{subfiles}
\begin{document}
\section{Introduction}
An example of modal analysis on a Timoshenko beam is given in section \ref{sec:existence:ModalAnalysis}. In this chapter, more general theory for modal analysis of te Timoshenko beam theory is discussed, as well as more examples of the application of the theory.

\section{Eigenvalue problem}\label{sec:Timo:EigenvalueProblem}
This section is a discussion of a systematic method to solve the eigenvalue problem for the Timoshenko beam theory. The article discussed is \cite{VV06}.

Consider a general eigenvalue problem for a Timoshenko beam Problem.
\subsubsection{General eigenvalue problem} \label{sssec:Timo:EigenvalueProblem}
Find functions $u$ and $\phi$ and a real number $\lambda$ satisfying the following equations
\begin{eqnarray}
-u'' + \phi' &=& \lambda u, \label{eq:Timo:EigenvalueProblem1}\\
-\alpha u' + \alpha\phi - \frac{1}{\gamma}\phi'' &=& \lambda\phi.\label{eq:Timo:EigenvalueProblem2}
\end{eqnarray}

Recall from section \ref{sec:1D_Model}, $u$ represents the transverse motion of the beam, $\phi$ is the angle of rotation of the cross-section of the beam, $\alpha$ and $\gamma$ are dimensionless constants defined in section 1.5 and $\lambda$ represents the eigenvalue. 

The authors of \cite{VV06} first derive a general solution for the system of ordinary differential equations \eqref{eq:Timo:EigenvalueProblem1} and \eqref{eq:Timo:EigenvalueProblem2}. In \cite{VV06} the authors show that for the Timoshenko models in this dissertation $\lambda$ non-negative. Assume that the solution is of the form $\langle u, \phi \rangle = e^{mx}\bar{w}$ where $m\in \mathbb{R}$ or $m\in \mathbb{C}$. Substitution into \eqref{eq:Timo:EigenvalueProblem1} and \eqref{eq:Timo:EigenvalueProblem2} yields:
\begin{eqnarray*}
-m^{2}e^{mx}w_{1}+me^{mx}w_{2}&=&\lambda e^{mx}w_{1},\\
-\frac{1}{\gamma}m^{2}e^{mx}w_{2}-\alpha me^{mx}w_{1}+\alpha e^{mx}w_{2}&=&\lambda e^{mx}w_{2}.
\end{eqnarray*}

From these equations it can be concluded that $e^{mx}\bar{w}$ is a solution of the system if and only if the pair $\langle m,\bar{w} \rangle$ is a solution of the linear system:

\begin{equation}
\begin{bmatrix}
-m^{2}-\lambda & m\\
-\alpha m & -\frac{1}{\gamma}m^{2}+(\alpha- \lambda)
\end{bmatrix}
\begin{bmatrix}
w_{1}\\w_{2}
\end{bmatrix}
=
\begin{bmatrix}
0\\0
\end{bmatrix}. \label{eq:Timo:SolutionEigenvalueProblem}
\end{equation}

Let $A$ denote the coefficient matrix of \eqref{eq:Timo:SolutionEigenvalueProblem}. For a nontrivial solution of $\bar{w}$, it is required that the determinant of matrix $A$ is $0$.
\begin{align*}
\textrm{det}(A)=m^{4}+\lambda(1+\gamma)m^{2}+\gamma\lambda(\lambda-\alpha)=0. \label{TE7}
\end{align*}

This equation is called the characteristic equation. The characteristic equation is quadratic with respect to $m^2$. The roots of the characteristic equation can be expressed as
\begin{align}
m^{2}=-\frac{1}{2}\lambda(1+\gamma)(1\pm\sqrt{\Delta}),
\end{align}
where
\begin{align}
\Delta=1-\frac{4\gamma}{(1+\gamma)^{2}}\left(1-\frac{\alpha}{\lambda}\right)=\frac{4\gamma}{(1+\gamma)^{2}}\frac{\alpha}{\lambda}+\frac{(1-\gamma)^{2}}{(1+\gamma)^{2}}.
\end{align}

It is clear that $\Delta>0$, since $\lambda$, $\alpha$ and $\gamma$ are all positive. Therefore $m^{2}$ will always be real, and have distinct roots.\\

Consider the case where $m^{2} = 0$, which occurs when $\lambda = \alpha$. Then the matrix A simplifies to
\begin{equation*}
\begin{bmatrix}
	\alpha & 0\\
	0&0
	\end{bmatrix}.
\end{equation*}

In  this case it is easy to find two linearly independent solutions:
\begin{eqnarray*}
\left[u(x) \ \phi(x)\right]^{T} = \left[0 \ 1 \right]^{T} \ \ \textrm{ and } \ \ \left[u(x) \ \phi(x)\right]^{T} = \left[1 \ \alpha x \right]^{T}.
\end{eqnarray*}



Now if $m^{2} \neq 0$, we consider the system $A\bar{w}=0$ with $w_{1} = m$ and $w_{2} = m^{2} + \lambda$. From \cite{VV06}, the parameters $\omega$, $\mu$ and $\theta$ are uniquely determined by $\lambda$.
\begin{eqnarray}
	\omega^2 & = & \frac{1}{2}\lambda(1+\gamma)(\Delta^{\frac{1}{2}}+1) \ \textrm{ for } \lambda >0\\
	\mu^2 & = & \frac{1}{2}\lambda(1+\gamma)(\Delta^{\frac{1}{2}}-1) \ \textrm{ for } \lambda < \alpha\\
	\theta^2 & = & \frac{1}{2}\lambda(1+\gamma)(1-\Delta^{\frac{1}{2}}) \ \textrm{ for } \lambda > \alpha
\end{eqnarray}

 The three cases $\lambda < \alpha$, $\lambda = \alpha$ and $\lambda > \alpha$ are considered separately by the authors of \cite{VV06}. The general solution from \cite{VV06} for each case is presented below.

\subsubsection{Case $\boldsymbol{\lambda<\alpha}$}

Denote the roots of (\ref{TE7}) by $\pm \mu$ and $\pm \omega i$. Thus the general solution is given by
\begin{eqnarray*}
\begin{bmatrix}
u(x)\\ \phi(x)
\end{bmatrix}
&=&
A
\begin{bmatrix}
\sinh (\mu x) \\ \frac{\lambda+\mu^{2}}{\mu}\cosh (\mu x)
\end{bmatrix}
+
B
\begin{bmatrix}
\cosh (\mu x) \\ \frac{\lambda+\mu^{2}}{\mu}\sinh (\mu x)
\end{bmatrix}
+
C
\begin{bmatrix}
\sin (\omega x) \\ -\frac{\lambda-\omega^{2}}{\omega}\cos (\omega x)
\end{bmatrix}\\
&&
+
D
\begin{bmatrix}
\cos (\omega x) \\ \frac{\lambda-\omega^{2}}{\omega}\sin (\omega x)
\end{bmatrix}.
\end{eqnarray*}

\subsubsection{Case $\boldsymbol{ \lambda=\alpha}$}

In this case the roots of (\ref{TE7}) are $0$ with multiplicity 2, and $\pm \omega i$. The general solution is
\begin{eqnarray*}
	\begin{bmatrix}
		u(x)\\ \phi(x)
	\end{bmatrix}
	=
	A
	\begin{bmatrix}
		0 \\1
	\end{bmatrix}
	+
	B
	\begin{bmatrix}
		1 \\ \alpha x
	\end{bmatrix}
	+
	C
	\begin{bmatrix}
		\sin (\omega x) \\ -\frac{\lambda-\omega^{2}}{\omega}\cos (\omega x)
	\end{bmatrix}
	+
	D
	\begin{bmatrix}
		\cos (\omega x) \\ \frac{\lambda-\omega^{2}}{\omega}\sin (\omega x)
	\end{bmatrix}.\label{T9}
\end{eqnarray*}

\subsubsection{Case $\boldsymbol{\lambda>\alpha}$}

All the roots of (\ref{TE7}) are complex. Denote them by $\pm \theta i$ and $\pm \omega i$. The general solution is
\begin{eqnarray*}
\begin{bmatrix}
u(x)\\ \phi(x)
\end{bmatrix}
&=&
A
\begin{bmatrix}
\sin (\theta  x) \\-\frac{\lambda-\theta^{2}}{\theta}\cos (\theta x)
\end{bmatrix}
+
B
\begin{bmatrix}
\cos (\theta x) \\ \frac{\lambda-\theta^{2}}{\theta}\sin (\theta x)
\end{bmatrix}
+
C
\begin{bmatrix}
\sin (\omega x) \\ -\frac{\lambda-\omega^{2}}{\omega}\cos (\omega x)
\end{bmatrix}\\
&&
+
D
\begin{bmatrix}
\cos (\omega x) \\ \frac{\lambda-\omega^{2}}{\omega}\sin (\omega x)
\end{bmatrix}.\label{T10}
\end{eqnarray*}

\subsubsection*{Strategy for determining the eigenvalues and eigenvectors}

The authors of \cite{VV06} provide a detailed example of the application of the above strategy to for determining the eigenvalues and eigenvectors. The example used is a pinned-pinned beam, the same model as in section \ref{subsec:Timoshenko_Modal_Analsis}.

The strategy can be summarized as follows:

From the general solutions for $u$ and $\phi$, the eigenvalues and eigenfunctions can be determined by imposing the boundary conditions at $x=0$ to reduce the solution space of four dimensions to a solution space of dimension at least two. 

The boundary condition at $x=1$ is then substituted which results in a homogeneous system of linear equations of the form
\begin{eqnarray*}
A \bar{b} = \bar{0}.
\end{eqnarray*} 

This system either has the zero solution or infinitely many solutions. To ensure the zero solution, the determinant of the matrix $A$ is set to zero. This equation $\textrm{det}(A) = 0$ is called the frequency equation.

The frequency equation has infinitely many solutions. Each of the solutions corresponds to a eigenvalue with a unique vector $\bar{b}$. The eigefunctions can then be obtained by substituting the vector $\bar{b}$ into the general solution for $u$ and $\phi$.

\subsubsection*{Pinned-pinned beam}

Returning back to the example of the pinned-pinned beam, we present some of the results from \cite{VV06}:

For $\lambda < \alpha$ the general solution reduces to $\langle u(x) , \phi(x) \rangle = \langle \sin(k \pi x), A_k \cos(k \pi x) \rangle$. 

For $\lambda = \alpha$, the general solution reduces to $\langle u(x) , \phi(x) \rangle = \langle 0, 1 \rangle$.

For $\lambda > \alpha$, the general solution also reduces to $\langle u(x) , \phi(x) \rangle = \langle \sin(k \pi x), A_k  \cos(k \pi x) \rangle$. 


A rigourous proof of the above results can be found in \cite{VV06}. These results also prove the Theorem 1 presented in section \ref{subsec:Timoshenko_Modal_Analsis}.


\end{document}
%obvisouly the solution space is 4 dimensional . bc at 0 makes it at most two.