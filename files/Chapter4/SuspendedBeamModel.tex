\documentclass[../../main.tex]{subfiles}
\begin{document}
\setcounter{chapter}{4}
\section{Suspended beam model}
The suspended Timoshenko beam model is referred to as Problem T-3 in this \ref{ssec:1D_Model:ModelProblems}. The beam is suspended at both endpoints by linear springs. The boundary conditions of Problem T-3 as given in section \ref{ssec:1D_Model:ModelProblems} can be rewritten by substituting the constitutive equations \eqref{eq:1D_Model:ConstitutiveEquations1D} and \eqref{eq:1D_Model:ConstitutiveEquations2D}.

	\[\partial_{x}w(0,\cdot) - \phi(0,\cdot) = kw(0,\cdot)  \ \ \ \textrm{and} \ \ \ \partial_{x}\phi(0,\cdot) = 0\]
	\[\partial_{x}w(1,\cdot) - \phi(1,\cdot) = -kw(0,\cdot) \ \  \ \textrm{and} \ \ \ \partial_{x}\phi(1,\cdot) = 0\]
	


The linear springs are allowed to take the weight of the beam and the system settles in a equilibrium state. The elongation of the springs from their natural length is denoted by $h$. The displacement from this equilibrium state should be considered.\\ 

Let $\langle w^* \ \phi^*\rangle$ represent the solution of the model problem, and $\langle w^e \ \phi^e \rangle$ the solution to the equilibrium problem. Define $\langle w \ \phi\rangle = \langle(w^*-w^e) \ \ (\phi^*-\phi^e)\rangle$ such that it represents the deviation of the beam from the equilibrium solution. Substitution verifies that $\langle w \ \phi\rangle$ satisfies the partial differential equation and boundary conditions.\\

If the beam is suspended by cables, in their loaded state they can be considered as linear springs. However if $w > h$ then the cables are no longer supporting the beam. The problem then becomes non-linear. Assume that the motion is small enough so that the problem remains linear.

\subsubsection{Variational problem}
Find $w,\phi$ such that for all $t>0$, $\langle w,\phi \rangle \in C[0,1]\times C[0,1]$ and
\begin{eqnarray*}
	\int_{0}^{1} \partial_{t}^{2} wv + \frac{1}{\alpha}\int_{0}^{1}  \partial_{t}^{2} \phi \psi &=& \int_{0}^{1} (\partial_{x}w - \phi)(\psi -v')  - \frac{1}{\beta}\int_{0}^{1}\partial_{x}\phi \psi'\\
	& & - kv(1)w(1,t) - kv(0)w(0,t)
\end{eqnarray*} for all $v,\psi \in C[0,1]$.\\

This model is required for explanation in Section 4.6.

\begin{comment}

\subsubsection{Modal Analysis}
The boundary conditions at $x = 0$ results in the following solutions for the constants of the general solution of the eigenvalue problem:
\begin{align}
	C  =  \frac{\omega}{\mu}A - k\frac{\omega}{\lambda} \left(1+\frac{\mu^{2}+\lambda}{\omega^2-\lambda}\right)B & \textrm{ and } \ D = \frac{ \mu^2+\lambda}{\omega^2-\lambda}B \ \ \ \textrm{ if } \lambda < \alpha \label{A1}\\
	C = \frac{\omega}{\lambda}A + k\frac{\omega}{\lambda}\left(1+\frac{\alpha}{\lambda - \omega^2}\right)B  \ & \textrm{ and } \ D = \frac{\alpha}{\omega^2-\lambda}B  \ \textrm{ if } \lambda = \alpha \label{A2}\\
	C=-\frac{\omega}{\theta}A+k\frac{\omega}{\lambda}\left(1-\frac{\theta^2-\lambda}{\omega^2-\lambda}\right)B  \ & \textrm{ and } \ D = -\frac{\theta^2-\lambda}{\omega^2 - \lambda}B  \ \textrm{ if } \lambda > \alpha \label{A3}
\end{align}

The boundary conditions at $x = 1$ gives the following homogeneous system of equations with the entries of the coefficient matrix given below.
\begin{align}
	\begin{bmatrix}
		M_{11}(\lambda) & M_{12}(\lambda)\\
		M_{21}(\lambda) & M_{22}(\lambda)
	\end{bmatrix}
	\begin{bmatrix}
		A\\
		B
	\end{bmatrix}
	= 
	\begin{bmatrix}
		0\\
		0
	\end{bmatrix}
	\label{Mat}
\end{align}

{If $\lambda < \alpha$}
\begin{align*}
&M_{11}(\lambda) = \sinh(\mu)(\lambda+\mu^2) + \frac{\omega(\lambda - \omega^2)}{\mu}\sin(\omega)&\\
&M_{12}(\lambda) = - k\frac{\omega(\mu^2+\omega^2)}{\lambda}\sin(\omega) + (\lambda+\mu^2)(\cosh(\mu)-\cos(\omega))& \\
&M_{21}(\lambda) = k\sinh(\mu) + k\frac{\omega}{\mu}\sin(\omega) +  \frac{\lambda}{\mu}(\cos(\omega) - \cosh(\mu))&\\
&M_{22}(\lambda) =\frac{\lambda^3+\lambda^2\mu^2-k^2\omega^4-k^2\mu^2\omega^2}{\lambda\omega(\lambda-\omega^2)}\sin(\omega) - \frac{\lambda }{\mu}\sinh(\mu) -k\frac{\lambda+\omega^2+2\mu^2}{(\lambda-\omega^2)}\cos(\omega) -k\cosh(\mu)&
\end{align*}


{If $\lambda = \alpha$}
\begin{align*}
&M_{11}(\lambda) = \frac{\omega\left(\lambda - \omega^2\right)}{\lambda}\sin(\omega)&\\
&M_{12}(\lambda)=\alpha-\alpha\cos(\omega)-k\frac{\omega}{\lambda}\left(\frac{\alpha}{\omega^2+\lambda}+1\right)(\lambda-\omega^2)\sin(\omega)&\\
&M_{21}(\lambda)=\frac{\omega^2}{\lambda}\cos(\omega)+\cos(\omega)\frac{\left(\lambda - \omega^2\right)}{\lambda} + k\sin(\omega)\frac{\omega}{\lambda} - 1 &\\
&M_{22}(\lambda) = k-\alpha-k\left(1-\frac{\alpha}{\lambda-\omega^2}-\frac{\alpha }{\omega^2+\lambda}\right)\cos(\omega)+\left(\frac{\alpha}{\omega}-k^2\frac{\omega}{\lambda}+\frac{\alpha\omega}{\lambda-\omega^2}-k^2\frac{\alpha\omega}{\lambda(\omega^2+\lambda)}\right)\sin(\omega)&
\end{align*}

{If $\lambda > \alpha$}
\begin{align*}
&M_{11}(\lambda) = \left(\lambda - \theta^2\right)\sin(\theta) - \frac{\omega  \left(\lambda - \omega^2\right)}{\theta}\sin(\omega)&\\
&M_{12}(\lambda) =  - k\frac{\omega(\omega^2- \theta^2)}{\lambda}\sin(\omega) + \left(\lambda - \omega^2\right)(\cos(\omega) - \cos(\theta))&\\
&M_{21}(\lambda)=k\frac{\omega}{\theta}\sin(\omega)-k\sin(\theta)-\frac{\lambda}{\theta}(\cos(\omega)-\cos(\theta))&\\
&M_{22}(\lambda)=\frac{\lambda^3-\lambda^2\theta^2-k^2\omega^4+k^2\omega^2\theta^2}{\lambda\omega(\lambda-\omega^2)}\sin(\omega)-\frac{\lambda}{\theta}\sin(\theta) + k\cos(\theta)-k\frac{\lambda-2\theta^2+\omega^2}{(\lambda-\omega^2)}\cos(\omega)&
\end{align*}

The frequency equations can be determined by simplifying the equation $$M_{11}(\lambda)M_{22}(\lambda)-M_{12}(\lambda)M_{21}(\lambda) = 0.$$
\end{comment}
\end{document}
