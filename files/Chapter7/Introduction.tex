\documentclass[../../main.tex]{subfiles}
\begin{document}
\section{Introduction}
This chapter is an investigation into the validity of a cantilever Timoshenko beam and a cantilever Reissner-Mindlin. Validity in this context refers to how well the solutions of a model compare to the solutions of a more realistic, higher-dimensional model. The application of the models are limited to beam-type and plate-type problems.

The chapter consists of three main parts.

Section 7.2 is a discussion of the work and a replication of numerical results of/ the article \cite{LVV09}. In this article, the authors compare a cantilever Timoshenko beam to a cantilever two-dimensional beam. The results of this section are also used as a reference for the rest of the chapter.

Section 7.3 is a extension of the article \cite{LVV09} to a comparison of a cantilever two-dimensional beam to a cantilever three-dimensional beam. This extension is mentioned by the authors of \cite{LVV09}. This is an investigation into the validity of a two-dimensional beam model, and is an intermediate step to validate the Timoshenko beam model against a three-dimensional beam model. A direct comparison is not considered due to the complexities involved, which is discussed in more detail in the numerical results.

Section 7.4 extends the work of \cite{LVV09} further to plate models. This extension follows the same idea as the previous sections, and originates from non-beam type behaviour observed in the three-dimensional beam models. In this section, a cantilever two-dimensional Reisner-Mindlin plate model is compared to a cantilever three-dimensional plate model.


\subsection{Global parameters}
All the models in this dissertation are assumed to be made of the same isotropic material and have a square cross-section. The parameters are as follows:

\begin{itemize}
    \item Elastic modulus ($G$): This is calculated using the formula $G = \frac{E}{2(1+\nu)}$, where $E$ is the modulus of elasticity and $\nu$ is Poisson's ratio. 

    \item Shear correction factor ($\kappa^2$): This is set to $5/6$, which is common for rectangular cross-sections.

    \item Poisson's ratio ($\nu$): This is set to $0.3$, a typical value for materials like steel used in engineering.
\end{itemize}

These paremeters are used in all the models, and their consistency helps to ensure that any differences in the results can be attributed to the model structures themselves, rather than variations in the material or geometric properties.

Since our models are all dimensionless, all of the beams and plates have a length of $\ell = 1$. We'll use $h$ to describe the height of the beams and plates, and $b$ to describe the width of the beams and plates.

\subsection{Model specific parameters}
The following parameters are not shared over all the models. These parameters can be unique to a model, or can be shared between some models but not all. These parameters mostly affect the geometry of the models.

\subsubsection{Timoshenko beam}
In the derivation of the Timoshenko beam model in Section \ref{subsec:equationsofmotion+constitutiveequation}, the parameter $\alpha$ is introduced.
\begin{eqnarray*}
	\alpha = \frac{A\ell^2}{I}
\end{eqnarray*}

The model is in a dimensionless form, therefore the length of the beam is $\ell = 1$. Since we assumed a square cross-section, the area of the cross-section can be calculated as $\displaystyle A = hb$. The area moment of inertia can also be calculated as $\displaystyle I = \frac{h^3b}{12}$.

Substituting these values into the equation for $\alpha$ gives the following relationship between the height of the beam and the parameter $\alpha$
\begin{eqnarray*}
	\alpha & = & \frac{12}{h^2},
\end{eqnarray*}
or equivalently,
\begin{eqnarray*}
	h & = & \sqrt{\frac{12}{\alpha}}.
\end{eqnarray*}

Using this parameter, the height of the beam can be set to any value. Some examples of values for $\alpha$ and the corresponding height of the beam that can be expected in this chapter, are given in Table \ref{tab:alphaheight} below.

\begin{table}[h]
	\centering
	\begin{tabular}{|c|c|}
		\hline
		$\alpha$ & $h$\\
		\hline
		$100$ & $\approx 0.35$ \\
		$300$ & $1/5$ \\
		$1200$ & $1/10$ \\
		$4800$ & $1/20$ \\
		$9600$ & $\approx 0.035$ \\
		$10800$ & $1/30$ \\
		\hline
	\end{tabular}
	\caption{Values of $\alpha$ and the corresponding height of the beam.}
	\label{tab:alphaheight}
\end{table}

\subsubsection{Two-dimensional beam}

For the two-dimensional beam, only the height parameter $h$ is required. The width of the beam is assumed to be finite, constant and small. 

\subsubsection{Three-dimensional beam}
The three-dimensional beam is the most complex model and has two parameters to consider, the height $h$ and the width $b$.

For beam type problems, one would assume that the width of the beam not much larger than the height of the beam. The numerical results will show that this guess is correct.

For plate type problems, the width of the plate is assumed to be much larger than the height of the plate. This assumption is also verified in the numerical results.

\subsubsection{Two-dimensional plate}
The two-dimensional plate model is the most different model from the rest of the models. The two-dimensional plate model is a Reissner-Mindlin plate model, and is derived in Section \ref{sec:reissnermindlinplate}. The model has two parameters, the height $h$ and the width $b$.

% Need to explain the two parameters.


\end{document}


This is the same idea as done in the article [LVV09]. To make the comparison between models of different dimensions, the eigenvalues and eigenfuctions of the models are considered. If the eigenvalues and eigenfunctions compare well, it can be assumed that the solution to the model problem would also compare well. The reason for this is explained in Section 1.5.5. And since the Fourier coefficients in the solution decreases very rapidly, only the first few eigenvalues and eigenfunctions need to be considered.\\

Section 6.2 looks at some of the results obtained in [LVV09]. The results for the Euler-Bernoulli beam theory is excluded a the article has sufficient results. The idea of Section 6.2 is to verify the conclusions obtained in [LVV09]. The results of the section also aggregate the work of the article by examining different values for the parameter $\alpha$.\\

Section 6.3 extends the work of [LVV09] by examining the comparison between a two-dimensional and three-dimensional cantilever beam model. The authors' opinion is that a comparison between a two-dimensional and three-dimensional model would be less problematic than comparing the one-dimensional (Timoshenko) beam model to a three-dimensional model. The two-dimensional model can be seen as an intermediate step.


\subsubsection{Parameters}
For all models, the parameters are chosen as $\nu = 0.3$, and $\kappa^2 = 5/6$. It is assumed that all models are made from the same isotropic material, hence $G = \frac{E}{2(1+\nu)}$. Since the beams have a rectangular cross-section, for the Timoshenko beam model the area moment of inertia can be calculated as
\begin{eqnarray*}
	I & = & \frac{h^3b}{12}.
\end{eqnarray*}
The dimensionless constant $\alpha$, given in Section 1.5.1, gives the following relationship between the beam's height, length and the value of $\alpha$:
\begin{eqnarray*}
	\frac{h}{\ell}  & = & \sqrt{\frac{12}{\alpha}}
\end{eqnarray*}
This equation show that the ratio of the height of the beam and the length is dependant on the value of the parameter $\alpha$. The parameter $\alpha$ only appears in the one-dimensional beam model. But for continuity will be used for all models in this chapter to determine the ratio of length to height of the models.

\subsubsection{Remark:}
\begin{itemize}
	\item[--] All results in this chapter are given to 3 significant digits.
	\item[--] ****b does not depend on $\alpha$
\end{itemize}

**********************Needs work******************\\
The comparisons in this Chapter are as follow:
\begin{itemize}
	\item[] Section 7.2: The Timoshenko cantilever beam (Problem T-2) and two-dimensional cantilever beam (Problem 2D-1).
	\item[] Section 7.3: The two-dimensional cantilever beam (Problem 2D-1) and three-dimensional cantilever beam(Problem 3D-1)
	\item[] Section 7.4: The Reisner-Mindlin cantilever plate (Problem P-1) and three-dimensional cantilever plate (Problem 3D-1)
\end{itemize}

For the first part of this chapter, the work of the authors of \cite{LVV09} are discussed. The authors of 




This chapter is a discussion and an extension of the work of the authors of \cite{LVV09}. In \cite{LVV09}, the authors examined the validity of a cantilever Timoshenko beam using a cantilever two-dimensional beam as a reference.\\

After the discussion of \cite{LVV09}, the work of the authors is extended to investigate the validity of a cantilever two-dimensional beam using a cantilever three-dimensional beam as a reference. Finally the validity of a cantilever Reissner-Mindlin plate is investigated using a cantilever three-dimensional plate model as reference. The structure and idea of the authors of \cite{LVV09} is used a guide throughout this chapter.\\

The chapter is divided into the following sections:\\
