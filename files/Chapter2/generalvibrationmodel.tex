\documentclass[../../main.tex]{subfiles}

\begin{document}

\section{Introduction} \label{sec:existence}
The models in this dissertation concern vibration of elastic bodies. These type of models have a similar variational form as the wave equation and are called second order hyperbolic type problems. This chapter discusses theory for the existence and uniqueness of solutions for these type of problems. The theoretical basis for modal analysis is also presented.

In this chapter we consider the work done by the authors of \cite{VV02} and \cite{VS18}. In these articles the authors prove the existence and uniqueness of solutions for general second order hyperbolic type problems. The problems in Chapter 1 are examples. The article \cite{VV02} concerns the case of a symmetric bilinear form $b$. The article \cite{VS18} extends this work where $b$ need not be symmetric. The article \cite{VV02} is sufficient for the models of this dissertation, while \cite{VS18} is only used where the notation is more convenient and to gain more insight.

Before the general theory is discussed, a model is presented to be used as an example to show the application of the theory. The model is a cantilever Timoshenko beam, denoted by Problem T-2 in Section \ref{ssec:1D_Model:ModelProblems}.

In Section \ref{ssec:1D_Model:VariationalForm}, the variational problem for the pinned-pinned beam is derived. The derivation of the cantilever beam model is similar, with a different test function space. Recall the variational problem for the cantilever beam Problem T-2V, with the test function space $\displaystyle T[0,1] = F_1[0,1] \times F_1[0,1]$. To determine the solvability of the problem, the weak variational problem is considered first.

To obtain the weak variational form of Problem T-2V some preparation is required. A natural setting for the problem is the product space $L^2(0,1)\times L^2(0,1)$, denoted by $X$. The inner product for $L^2(0,1)$ yields the inner product $\displaystyle (x_1,y_1) + (x_2,y_2)$ for $X$. The idea is to replace the pair $\langle w , \phi \rangle$ by a function $u$ ``of time only" and values $u(t)$ in the space $X$. \label{sym:Ln}

Define a function u with domain $J$, an interval of real numbers. The range of $u$ is contained in $X$:
\begin{eqnarray}
	u(t)(x) = \langle w(x,t), \phi(x,t) \rangle. \label{def_of_u}
\end{eqnarray}

A derivative $u'$ for $u$ may be defined by
\begin{eqnarray*}
	||\left(h^{-1} u(t+h)-u(t)\right)-u'(t)||_X \rightarrow 0.
\end{eqnarray*} Then $u''$ is defined by $u'' = (u')'$. Sometimes the derivatives are denoted by $\dot{u}$ and $\ddot{u}$.\\

Using the bilinear forms, $b$ and $c$ from Section 1.4, Problem T-2V can be rewritten in the following form
\begin{eqnarray}
	c(\ddot{u}(t),v) + b(u(t),v) & = & (Q(t),v), \label{eq:existence:PreWV}
\end{eqnarray} for each $v \in T[0,1]$.

To apply the theory from \cite{VV02}, complete function spaces are required. It is known that $L^2(0,1)$ is a complete function space and hence $X$ is complete.

The Sobolev space $H^1(0,1)$ is all the functions in $L^2(0,1)$ with at least a first-order weak derivative. $H^1(0,1)$ is complete (see Appendix A).\label{sym:Hn}

The test functions $T[0,1]$ do not form a complete space. Let $V(0,1)$ be the closure of {$F_1(0,1)$} in $H^1(0,1)$ (see Section 1.4.4). It follows that $V(0,1)$ is complete and the product space {$V(0,1)\times V(0,1)$}, denoted by $V$, is complete. It is called the energy space.

Using these complete product spaces and \eqref{eq:existence:PreWV}, the weak variational problem for the cantilever Timoshenko beam is defined as Problem T-2W.

\subsubsection{Problem T-2W}\label{sssec:existence:ProblemT2W}
Find a function $u$ such that $\forall \ t \in J$, $u(t) \in V$, $\ddot{u}(t) \in W$ and
\begin{eqnarray*}
	c(\ddot{u}(t),v) + b(u(t),v) & = & (Q(t),v)
\end{eqnarray*}
for all $v\in V$. The initial conditions $u(0) = u_0$ and $\dot{u}(0)=u_1$ must be specified.

The space X equipped with the inner-product $c$ is denoted by $W$. $W$ is called the inertia space and in Section 2.3 it is shown that $c$ is an inner-product for $W$.

In the next section, the theory of the article \cite{VV02} is presented. In Section 2.3 the theory is applied to Problem T-2W.

\section{Existence and uniqueness of solutions}\label{sec:existence:ExistenceAndUniqueness}
The general weak variational problem is studied in this section. The following theory is from the article \cite{VV02}. To start, some notation is given, as well as the relations between various Hilbert spaces. The necessary assumptions are also stated.\\

As mentioned before, ideas and notation from \cite{VS18} are also used.

\subsection{The variational approach}\label{ssec:existence:VariationlApproach}
Let $V$, $W$ and $X$ be real Hilbert spaces such that $W$ is a linear subspace of $X$, and $V$ is a linear subspace of $W$, i.e. $V \subset W \subset X$.
\begin{itemize}
	\item[] X is the global space with inner product $(\cdot,\cdot)_X$ and the induced norm $||\cdot||_X$.
	\item[] W is the inertia space with inner product $(\cdot,\cdot)_W$ and the induced norm $||\cdot||_W$.
	\item[] V is the energy space with inner product $(\cdot,\cdot)_V$ and the induced norm $||\cdot||_V$.
\end{itemize} \label{sym:Xspace}\label{sym:Wspace}\label{sym:Vspace}\label{sym:norm} \label{sym:innerproduct}

The following notation is important for the theory that follows.

Let $J$ be a interval of real numbers containing zero. It can have one of the following forms $[0,\ T)$, $[0,\ \infty)$ or an arbitrary open interval. For any function $u$ on the interval $J$ and range in a Hilbert space Z, derivatives can be defined. A derivative $u'$ for $u$ may be defined by
\begin{eqnarray*}
	||(h^{-1} u(t+h)-u(t))-u'(t)||_Z \rightarrow 0.
\end{eqnarray*} Then $u''$ is defined by $u'' = (u')'$.


\textbf{Notation}
\begin{itemize}
	\item[] $u\in C(J,Z)$ if $u$ is continuous on $J$ with respect to the norm of $Z$;

	\item[] $u'(t)\in Z$ if $u$ is differentiable with respect to the norm of $Z$;

	\item[] $u\in C^{k}(J,Z)$ if $u^{(k)}\in C(J,Z)$ .
\end{itemize}

\textbf{Remark} For Problem T-2, $Z$ can be $X$, $W$ or $V$.

Let $a$, $b$ and $c$ be bilinear forms where $a$ and $b$ are defined on $V$ and $c$ is defined on $W$. For the models in this dissertation, the bilinear forms $b$ and $c$ are symmetric. Furthermore, $b(\cdot, \cdot) = (\cdot, \cdot)_V$ and $c(\cdot, \cdot) = (\cdot, \cdot)_W$.


\subsubsection*{Problem GVar}\label{sssec:existence:ProblemGVar}
Given a function $f:J\rightarrow X$, find a function $u\in C(J,\ X)$ such that $u'$ is continuous at $0$ with respect to $\Vert \cdot \Vert_{W}$ and for each $t\in J,\ u(t)\in V,\ u'(t) \in V,\ u''(t)\in W$ and
\begin{eqnarray}
	c(u''(t),v)+a(u'(t),v)+b(u(t),v)= (f(t),v)_{X} \ \ \ \ \textrm{for each} \ v \in V, \label{eq:existence:ProblemGVar}
\end{eqnarray}
while $u(0)=u_{0},\ \ \ u'(0)=u_{1}$.\label{sym:a}

This general variational form is applicable to all the models in this dissertation. In \cite{VV02}, a more general form is given that includes the possibility for damping terms in the models. Damping is not considered in this dissertation.

\subsubsection*{Assumptions} \label{sssec:existence:Assumptions}
The following assumptions are made in \cite{VV02} for the existence results.
\begin{itemize}
	\item[] \textbf{A1} - $V$ is dense in $W$ and $W$ is dense in $X$.

	\item[] \textbf{A2} - There exists a positive constant $C_{W}$ such that $\Vert w\Vert_{X} \leq C_{W}\Vert w\Vert_{W}$ for each $ w\in W$.

	\item[] \textbf{A3} - There exists a positive constant $C_{V}$ such that $\Vert v\Vert_{W} \leq C_{V}\Vert v\Vert_{V}$ for each $v \in V$.

	\item[] \textbf{A4} - The bilinear form $a$ is non-negative, symmetric and bounded on $V$, i.e. there exists a positive constant $K_a$ such that for $\displaystyle u,v \in V$, \[|a(u,v)| \leq K_a\Vert u \Vert_V \Vert v \Vert_V.\]
\end{itemize}


In general, the bilinear form $a$ in \textbf{A4} is defined on $V$. However it is possible for $a$ to be defined on the space $W$ and bounded by the norm $||\cdot||_W$, i.e. there exists a $k > 0$ so that
\begin{eqnarray}
	|a(u,v)| \leq k ||u||_W||v||_W \ \ \ \ \textrm{ for all } \ u,\ v \in W. \label{weak_damping}
\end{eqnarray} This is called \textbf{weak damping}. Note that \eqref{weak_damping} is trivially satisfied if $a = 0$.



\subsection{Main results for existence and uniqueness}\label{ssec:existence:MainResultsExistence}
To start, the main results of \cite{VV02} are presented. There are three existence theorems but Theorem 2 is important for this dissertation.

\newtheorem{Thmx}{Theorem}
\begin{Thmx}[{Main Result}]
	Suppose assumptions \textbf{A1}-\textbf{A4} hold. If, for $u_0 \in V$ and $u_1 \in V$, there exists some $y \in W$ such that
	\begin{eqnarray}
		b(u_0,v) + a(u_1,v) = c(y,v) \ \ \ \ \textrm{ for all } \ v \in V, \label{bilinear_equation}
	\end{eqnarray}
	then for each $f \in C^1(J,X)$, there exists a unique solution $u \in C^1(J,V)\cap C^2(J,W)$ for Problem GVar.
\end{Thmx}

This theorem allows for a solution of the abstract variational problem Problem GVar, if the assumptions \textbf{A1}-\textbf{A4} is satisfied and the initial values $u_0$ and $u_1$ are admissible. It is not always easy to verify that \eqref{bilinear_equation} is satisfied.


In \cite{VV02} the authors consider a special case of weak damping. Define a space $E_b \subset V$ where \[ E_b = \left\{x \in V \ | \ \textrm{there exists a } y \in W \textrm{ such that } c(y,v) = b(x,v) \textrm{ for all } v \in V\right\}.\] It is proved in \cite{VV02}  that condition (2.2.3) is satisfied if $u_1 \in V$ and $u_0 \in E_b$ (the pair $u_0$, $u_1$ is admissible).


\newtheorem{Thm2n}[Thmx]{Theorem}
\begin{Thm2n}[Weak Damping]
	If $a$ is bounded with respect to the norm in $W$, then there exists a unique solution $u\in C^1(J,V)\cap C^2(J,W)$ for Problem GVar for each $u_0 \in E_b$. each $u_1 \in V$ and each $f\in C^1(J,X)$.
\end{Thm2n}

\textbf{Remark} In general, the bilinear form $a$ is non-negative. However it is possible for $a$ to be positive definite on the space $V$ with respect to the norm $||\cdot||_V$, i.e. there exists a $c_a > 0$ so that
\begin{eqnarray*}
	a(v,v) \geq c_a ||v||^2_V \ \ \ \ \textrm{ for all } \ v \in V.
\end{eqnarray*} This is called strong damping. It is not considered in this dissertation.

\newtheorem{Thm3n}[Thmx]{Theorem}
\begin{Thm3n}[Strong Damping]
	If $a$ is positive definite on $V$, there exists a unique solution $u \in C^1([0,\infty),V)\cap C^2((0,\infty),W)$ for Problem GVar for any $u_0 \in V$, $u_1 \in W$ and any $f$ which is Lipschitz on $V$. If $f=0$, then $u \in C^1([0,\infty),V)\cap C^2([0,\infty),W)\cap C^\infty((0,\infty),V)$.
\end{Thm3n} \label{sym:Cinfty}

As mentioned, in the models of this dissertation, the bilinear form $a$ is identically zero. This automatically satisfies the weak damping condition and hence Theorem 2 may be applied.


\subsection{First order system} \label{ssec:existence:AbstractDifferentialEquation}

To prove the main results in \cite{VV02}, the authors introduce an equivalent first order system.\\

This variational form is rewritten as a first order system of differential equations. Let $y(t) = u'(t)$, then \[ c(y'(t),v)+a(y(t),v)+b(u(t),v)= (f(t),v)_{X}.\] To make this precise, a Hilbert space H is defined by $H:=V\times W$. For $x \in H$, $x = \langle x_1, x_2 \rangle$ with $x_1 \in V, x_2 \in W$. An inner product on $H$ is defined by
\begin{eqnarray*}
	(x,y)_H := b(x_1,y_1) + c(x_1,y_1) \ \textrm{for all} \ x,y \in H.
\end{eqnarray*}

Then the authors define an operator $\Lambda$ as a mapping on $H$ by $\Lambda y = -x$ when $-x_2 = y_1$ and $x_1 \in V$ such that
\begin{eqnarray}
	b(x_1,v) + a(x_2,v) = c(y_2,v) \ \textrm{ for each } \ v \in V. \label{author_mapping}
\end{eqnarray}
The operator $A$ is defined in \cite{VV02} as $A = \Lambda^{-1}$ with $D(A) = \mathcal{R}(\Lambda)$.\\

From a result in \cite{VV02}, $x \in D(A)$ if and only if $x_1 \in V$, $x_2 \in V$ and there exists a $z \in W$ such that $b(x_1,v) + a(x_2,v) = c(z,v)$ for all $v \in V$. Furthermore $y = Ax$ if $y_1 = -x_2$ and \[b(x_1,v) + a(x_2,v) = c(y_2,v)\  \textrm{ for all } v \in V.\]

This operator is used to rewrite the equation \eqref{eq:existence:ProblemGVar} of Problem GVar as a first order differential equation in the form
\begin{eqnarray}
	x' & = & Ax + f. \label{Cauchy}
\end{eqnarray}

To be more precise, the following problem is introduced.


\subsubsection*{Problem IVP}\label{sssec:existence:ProblemIVP}
Given a function $F:J\rightarrow H$, find a function $U\in C(J,\ H)$ such that for each $t\in J$, $U(t) \in D(A)$, $U(t) \in H$ and
\begin{eqnarray*}
	U(t)' & = & AU(t) + F(t),\\
	U(0) & = & U_0.
\end{eqnarray*}

To link Problem IVP and Problem GVar, consider the following lemma from \cite{VV02}.

\newtheorem*{Prop}{Lemma 1}
\begin{Prop}
	Suppose $F(t) = \langle 0, f(t) \rangle$ for each $t\in J$.
	\begin{itemize}
		\item[a)] If $u$ is a solution of Problem GVar, then $U = \langle u, u' \rangle$ is a solution of Problem IVP, with $U_0 = \langle u_0, u_1 \rangle$.
		\item[b)] If $U$ is a solution of Problem IVP with $U_0 = \langle u_0, u_1 \rangle$, then the first component $u = U_1$ of $U$ is a solution for Problem GVar.
	\end{itemize}
\end{Prop}

Semi-group theory is used in \cite{VV02} to investigate the solvability of Problem IVP and obtain a solution for Problem IVP. It is also important to mention that in \cite{VV02}, the authors provide the necessary result that shows the function $F$ is uniquely defined by $f$.

As mentioned before, if $a = 0$ then the inequality in \eqref{weak_damping} hold trivially. However, the operator A is defined by the operator $\Lambda$ and in the definition of $\Lambda$, the form $a$ is used (see \eqref{author_mapping}). It is proved in \cite{VV02} that equation \eqref{author_mapping} is uniquely solvable and hence $\Lambda$ is well defined. The proof remains unchanged as presented for $a$ identically zero.

To apply Theorem 2, the admissible initial conditions are $u_0 \in E_b$ and $u_1 \in V$. As mentioned above, this implies $\langle u_0, u_1 \rangle \in D(A)$. From semigroup theory, it follows that $U(t) \in D(A)$ for each $t$. Therefore $u(t) \in E_b$ and $u'(t) \in V$ for each $t$.

Using the assumptions \textbf{A1}-\textbf{A4}, it is proved in \cite{VV02} that the linear operator $A$ is an infinitesimal generator of a $C_0$-semigroup of contractions and the domain of $A$ is dense in $H$ (see Section \ref{sec:existence:ModalAnalysis}).

\section{Application: Timoshenko beam model}\label{sec:existence:Application}
In this section the theory of \cite{VV02} (presented in Section \ref{sec:existence:ExistenceAndUniqueness}) is applied to Problem T-2W. This is an continuation of the example from Section \ref{sec:existence}.

Recall the spaces defined in Section \ref{sec:existence}:
\begin{itemize}
	\item[] $X = L^2(0,1)\times L^2(0,1)$ with inner product $(\cdot,\cdot)_X$ and induced norm $||\cdot||_X$.
	\item[] $W = X$ with inner product $c$ and induced norm $||\cdot||_W$.
	\item[] $V  = V(0,1)\times V(0,1)$ with inner product $b$ and induced norm $||\cdot||_V$.
\end{itemize}

To apply the theory, Problem T-2W must satisfy assumptions \textbf{A1}-\textbf{A4} and the initial values must be admissible. We must show that the assumptions are satisfied.

To prove \textbf{Assumption A1}, observe that $W$ is dense in $X$. (The set $W = X$).\\

Let $u \in C_0^\infty(0,1)$. Then $u(0) = u(1) = 0$ and therefore $u \in T(0,1)$ and $C_0^\infty(0,1) \subset T(0,1)$. Recall that $V(0,1)$ is the closure of $T(0,1)$ in $H^1(0,1)$. So it follows that $C_0^\infty(0,1) \subset V(0,1) \subset H^1(0,1)$. And since $C_0^\infty(0,1)$ is dense in $L^2(0,1)$, both $V(0,1)$ and $H^1(0,1)$ are dense in $L^2(0,1)$. Therefore $V(0,1)\times V(0,1)$ is dense in $X = L^2(0,1) \times L^2(0,1)$.\label{sym:Cinftyzero} 

Consider \textbf{Assumption A2}. From the definition of the bilinear form $c$,
\begin{eqnarray*}
	c(f,f) & = & \int_0^1  (f_1)^2 + \frac{1}{\alpha}\int_0^1  (f_2)^2 = ||f_1||^2 + \frac{1}{\alpha}||f_2||^2.\\
\end{eqnarray*}
From this and the definition of the $X$ norm, the following inequalities can be obtained:
\begin{eqnarray*}
	\min\left\{1, \frac{1}{\alpha}\right\} ||f||_{X}^2 \leq	c(f,f)	\leq \max\left\{1, \frac{1}{\alpha}\right\} ||f||_{X}^2.\\
\end{eqnarray*}

Since $c$ is a bilinear form and by the inequality above, $c$ is an inner-product for $W$. The norm of $W$ is defined as $||\cdot||_W = \sqrt{c(\cdot,\cdot)}$. Let $C_1 = \min\left\{1, \frac{1}{\alpha}\right\}$ and $C_2 = \max\left\{1, \frac{1}{\alpha}\right\}$. Then
\begin{eqnarray}
	C_1||x||_X \leq ||x||_W \leq C_2||x||_X \label{C}
\end{eqnarray} for all $x \in X$.

To prove \textbf{Assumption A3}, some preparation is required. Consider the following proposition for a Poincar\'e-Type inequality for the one-dimensional case.
\newtheorem*{Poincare}{Proposition 1}
\begin{Poincare}
	Suppose that $f \in C^1[a,b]$, and $f(a) = 0$. Then\\ $\displaystyle ||f|| \leq (b-a)||f'||$.
\end{Poincare}
\begin{proof}
	Let $f \in C^1(a,b)$ such that $f(a) = 0$. By the Fundamental Theorem of Calculus,
	\begin{eqnarray*}
		|f(x)| = \left|\int_a^x f'(s) \ ds\right| \leq \int_a^x |f'(s)| \ ds.
	\end{eqnarray*}
	Since $x$ is arbitrary,
	\begin{eqnarray*}
		||f||_{\sup} \leq \int_a^b |f'(s)| ds.
	\end{eqnarray*}
	Also for $f,g \in C^1(a,b)$,
	\begin{eqnarray}
		\left| \int_a^b fg \right| & \leq & ||f||\ ||g|| \ \ \ \ \textrm{ by Cauchy-Swartz Inequality.} \label{CSI}
	\end{eqnarray}
	Choose $g = 1$ then since $||f|| \leq ||f||_{\sup} \sqrt{b-a}$, the result follows.
\end{proof}

\newtheorem*{Poincare_2}{Corollary}
\begin{Poincare_2}
	Suppose that $f \in H^1(a,b)$, and $f(a) = 0$. Then $||f|| \leq (b-a)||f'||$.
\end{Poincare_2}
\begin{proof}
	There exists a sequence $(g_n) \subset C^1[a,b]$ such that $||g_n -f|| \rightarrow 0$ and $||g'_n-f'||\rightarrow 0$.

	Therefore $\displaystyle \frac{||f||}{||f'||} = \lim_{n\ \rightarrow \infty} \frac{||g_n||}{||g'_n||} \leq (b-a)$.
\end{proof}

\newtheorem{Theorem_1}{Theorem}
\begin{Theorem_1}
There exists a positive constant $C_{V}$ such that $\Vert v\Vert_{W} \leq C_{V}\Vert v\Vert_{V}$ for each $v \in V$.
\end{Theorem_1}

\begin{proof}
	Let $f \in V$, then $f_1(0) = f_2(0) = 0$. Following from the corollary, $||f_1|| \leq ||f_1'||$ and $||f_2|| \leq ||f_2'||$. Therefore,
	\begin{eqnarray*}
		||f_1'|| & = & ||f_1' - f_2 + f_2||,\\
				 & \leq & ||f_1' - f_2|| + ||f_2||,\\
				 & \leq & ||f_1' - f_2|| + ||f_2'||.
	\end{eqnarray*}

	It follows that $||f_1'||^2 \leq 2||f_1' - f_2||^2 + 2||f_2'||^2$, since $\displaystyle (a+b)^2 \leq 2a^2 + 2b^2$. Hence, from the definition of the norm $||\cdot||_X$, and again from the corollary,
	\begin{eqnarray*}
		||f||_X^2 & = & ||f_1||^2 + ||f_2||^2,\\
				  & \leq & ||f_1'||^2 + ||f_2'||^2,\\
				  & \leq & 2||f_1' - f_2||^2 + 3||f_2'||^2.
	\end{eqnarray*}

	It follows from the inequality \eqref{C} that
	\begin{eqnarray}
		\Vert f \Vert_{W}^2 \leq 2C_2||f_1'-f_2||^2 + 3C_2||f_2'||^2. \label{f}
	\end{eqnarray}

	But we also have
	\begin{eqnarray}
		b(f,f) & = &   \int_0^1 (f_1'-f_2)^2 + \frac{1}{\beta}\int_0^1  (f_2')^2, \nonumber\\
		& = & ||f_1' - f_2||^2 + \frac{1}{\beta}||f_2'||^2, \nonumber\\
		& \geq & \min\left\{1 ,\ \frac{1}{\beta} \right\} \left( ||f_1' - f_2||^2 + ||f_2'||^2\right). \label{b}
	\end{eqnarray}

	Now combine the inequalities \eqref{f} and \eqref{b}. There exists a positive constant $C_V$ such that
	\begin{eqnarray*}
		||f||_W^2 \leq C_V b(f,f),
	\end{eqnarray*} for all $f \in V$.
\end{proof}


Consider \textbf{Assumption A4}. In Problem T-2W, it is clear that the bilinear form $a$ is identically zero. As mentioned before, if $a = 0$, the weak damping as well as assumption \textbf{A4} is satisfied. Also, as explained in Section \ref{sec:existence:ExistenceAndUniqueness} the constuction of the operator $A$ is not affected.


\section{Modal analysis}\label{sec:existence:ModalAnalysis}
\subsection{Timoshenko beam}\label{subsec:Timoshenko_Modal_Analsis}
To understand what is meant by modal analysis, it is in the first place necessary to consider modes of vibration. And to understand how modal analysis is applied, a Timoshenko beam is used as an example. Specifically, Problem T-1 as the boundary conditions makes it easy to illustrate the idea of modal analysis. A detailed approach to modal analysis for the Timoshenko beam theory is discussed in Chapter 4.


Consider the partial differential equations of a pinned-pinned Timoshenko beam, obtained by substituting the constitutive equations \eqref{eq:1D_Model:ConstitutiveEquations1D} and \eqref{eq:1D_Model:ConstitutiveEquations2D} into the equations of motion \eqref{eq:1D_Model:EquationOfMotion1D} and \eqref{eq:1D_Model:EquationOfMotion2D}:
\begin{eqnarray}
	  \partial^2_{x} w - \partial_{x}\phi &=& \partial^{2}_{t} w, \label{eq:1D_Model:ModalPDE1}\\
	  \alpha\partial_{x} w - \alpha\phi + \frac{1}{\gamma}\partial^2_{x}\phi &=&  \partial^{2}_{t} \phi,\label{eq:1D_Model:ModalPDE2}
\end{eqnarray}
with boundary conditions
\begin{eqnarray*} 
	w(0,t) = 0, & &	\	 w(1,t) = 0, \label{eq:1D_Model:ModalBC1}\\
	M(0,t) = 0, & &	\	M(1,t) = 0. \label{eq:1D_Model:ModalBC2}
\end{eqnarray*}
The boundary conditions of the moments can be written in terms of $\phi$ using the constitutive equation \eqref{eq:1D_Model:ConstitutiveEquations1D}:
\begin{eqnarray*}
	\partial_{x}\phi(0,t) = 0, & & \partial_{x}\phi(1,t) = 0. \label{MA_4}
\end{eqnarray*}
Consider a trial solution $w(x,t) = T(t)\tilde{w}(x)$ and $\phi(x,t) = T(t)\tilde{\phi}(x)$. Substituting these trial solutions in \eqref{eq:1D_Model:ModalPDE1} and \eqref{eq:1D_Model:ModalPDE2} yields
\begin{eqnarray*}
	T(t)\tilde{w}''(x) - T(t)\tilde{\phi}'(x) &=& T''(t)\tilde{w}(x),\\
	  \alpha T(t)\tilde{w}'(x) - \alpha T(t)\tilde{\phi}(x) + \frac{1}{\gamma}T(t)\tilde{\phi}''(x) &=&  T''(t)\tilde{\phi}(x).
\end{eqnarray*}
Dividing by $T(t)$,
\begin{eqnarray*}
	\tilde{w}''(x) - \tilde{\phi}'(x) &=& \frac{T''(t)}{T(t)}\tilde{w}(x),\\
	  \alpha \tilde{w}'(x) - \alpha \tilde{\phi}(x) + \frac{1}{\gamma}\tilde{\phi}''(x) &=&  \frac{T''(t)}{T(t)}\tilde{\phi}(x).
\end{eqnarray*}

Therefore $\displaystyle \frac{T''(t)}{T(t)}$ is constant. Suppose $\displaystyle \frac{T''(t)}{T(t)} = -\lambda$. This can also be written as the following differential equation,
\begin{eqnarray}
	\ddot{T} + \lambda T = 0. \label{eq:1D_Model:ModalAnalysisT}
\end{eqnarray}

To determine if such a number $\lambda$ exists, consider the following eigenvalue problem.
\subsubsection*{Problem T-1E}\label{sssec:1D_Model:Problem:T1E}
Find $\lambda \in R$ and functions $\tilde{w}$ and $\tilde{\phi}$ such that
\begin{eqnarray*}
	-\tilde{w}'' + \tilde{\phi}' &=& \lambda \tilde{w}, \label{eq:1D_Model:ProblemT1E1}\\
	-\alpha \tilde{w}' + \alpha\tilde{\phi} - \frac{1}{\gamma}\tilde{\phi}'' &=& \lambda \tilde{\phi}, \label{eq:1D_Model:ProblemT1E2}
\end{eqnarray*} with boundary conditions $\tilde{w}(0) = \tilde{w}(1) = 0$ and $\tilde{\phi}'(0) = \tilde{\phi}'(1) = 0$.

Solutions to Problem T-1E is a pair of functions $\langle \tilde{w}, \tilde{\phi} \rangle$ called the eigenfunction with corresponding eigenvalue $\lambda$. Substitution of $\tilde{w} = \sin(k\pi x)$ and $\tilde{\phi} = \cos(k \pi x)$ show that they are solutions of the differential equations and satisfy the boundary conditions.

Clearly the eigenvalue problem has infinitely many solutions $\langle \tilde{w}_n, \tilde{\phi}_n \rangle$ and corresponding eigenvalues $\lambda_n$. Another solution is the vector $\langle \tilde{w}, \tilde{\phi} \rangle= \langle 0, 1 \rangle$ which also satisfies the eigenvalue problem with boundary conditions.

A systematic approach to obtain the eigenvalues and eigenfunctions is given in \cite{VV06}. This is discussed in more detail in Section \ref{sec:Timo:EigenvalueProblem}. To verify the trial solutions for Problem T-1E, the following theorem can be derived from \cite{VV06}.

\newtheorem{ThmVV06}{Theorem}
\begin{ThmVV06}
	If $\langle u, \phi\rangle$ is a non-constant eigenfunction of Problem T-1E, then $\langle u, \phi\rangle = \langle\sin k \pi x, A_k \cos k \pi x\rangle$ which satisfies the boundary conditions. $A_k$ is a constant depending on the integer $k$ and the eigenvalue $\lambda_k$. If $\langle u, \phi\rangle$ is a constant eigenfunction of Problem T1-E, then  $\langle u, \phi\rangle = \langle 0, 1 \rangle$. Also all eigenvalues less than $\alpha$ are simple eigenvalues.
\end{ThmVV06}

\begin{proof}
	See Section 4.2.
\end{proof}

Corresponding to this sequence of eigenfunctions, we have a sequence of solutions $(T_n(t))$:
\begin{eqnarray}
	T_n(t)=A_n\cos(\sqrt{\lambda_n}t)+B_n\sin(\sqrt{\lambda_n}t), \label{eq:1D_Model:ModalAnalysisTn}
\end{eqnarray}
(with $A_n$ and $B_n$ arbitrary constants) for the ordinary differential equation \eqref{eq:1D_Model:ModalAnalysisT}.

Combining the solutions of the eigenvalue problem with the solution \eqref{eq:1D_Model:ModalAnalysisTn} yields:
\begin{eqnarray*}
	w_n(x,t) & = & T_n(t)\tilde{w}_n(t), \\
	\phi_n(x,t) & = & T_n(t)\tilde{\phi}_n(x).
\end{eqnarray*} 
Substitution show that these solutions do indeed satisfy the differential equations \eqref{eq:1D_Model:ModalPDE1} and \eqref{eq:1D_Model:ModalPDE2} with the boundary conditions. These solutions are the modal solutions and they are clearly periodic with natural angular frequencies $\sqrt{\lambda_n}$ (and consequently the natural frequencies are $\frac{\sqrt{\lambda_n}}{2 \pi})$.

Therefore the formal series solution for Problem T-1 is
\begin{eqnarray*}
	\langle u, \phi \rangle = \sum_{n=1}^{\infty} T_n(t)\langle u_n(x), \phi_n(x)\rangle.
\end{eqnarray*}

The variational eigenvalue problem for Problem T-1E can be obtained using similar steps to how Problem T-1V is obtained. This variational eigenvalue problem is referred to as Problem T-1EV

\subsubsection*{Problem T-1EV}\label{sssec:1D_Model:ProblemT1EV}
Find a pair of functions $\tilde{w}$ and $\tilde{\phi}$ such that for all $t >0$, $\langle \tilde{w}\ \tilde{\phi}\rangle \in  T[0,1]$ and satisfying the equations
\begin{eqnarray*}
	 -\int_0^1 \tilde{w}'v'  + \int_0^1 \tilde{\phi} v' &=& \int_0^1 \lambda \tilde{w} v ,\\
	 -\int_0^1\alpha \tilde{w}' \psi + \int_0^1\alpha \tilde{\phi} \psi - \int_0^1\frac{1}{\gamma}\tilde{\phi}' \psi' &=& \int_0^1 \lambda \tilde{\phi} \psi .
\end{eqnarray*} for each $\langle v, \ \psi\rangle \in T[0,1]$. \label{sym:lambda}


\subsection{General vibration problem}
This section is a discussion of the general vibration problem GVar. The discussion will refer to the article \cite{CVV18}. In this article, the authors take damping into consideration, which is not covered in this dissertation. 

Consider Problem GVar defined in Section \ref{ssec:existence:VariationlApproach}. Assume there is no damping and no forcing. 

\subsubsection*{Problem GVar}\label{sssec:existence:ProblemGVar}
Find a function $u \in C(J,\ X)$ such that $u'$ is continuous at $0$ with respect to $\Vert \cdot \Vert_{W}$, and for each $t \in J$, $u(t) \in V$, $u'(t) \in V$, $u''(t) \in W$, satisfying
\begin{eqnarray}
c(u''(t),v)+b(u(t),v) = 0 \ \ \ \ \textrm{for each} \ v \in V, \label{eq:existence:ProblemGVarHom}
\end{eqnarray}
with initial conditions $u(0) = u_0$ and $u'(0) = u_1$.

To try and find a solution to this problem, consider a trial solution $u(t) = T(t)x$ with $x \in V$ and $x \neq 0$. Substituting this trial solution into \eqref{eq:existence:ProblemGVarHom} results in
\begin{eqnarray*}
	b(T(t)x,v) = - c(T''(t)x,v).  \label{eq:existence:ProblemGVarHom:Substitution}
\end{eqnarray*}

Due to the linearity of the bilinear form, this equation can be rewritten as
\begin{eqnarray*}
	T(t)b(x,v) = - T''(t)c(x,v).
\end{eqnarray*}

Dividing both sides by $T(t)$ gives
\begin{eqnarray*}
	b(x,v) = - \frac{T''(t)}{T(t)}c(x,v).
\end{eqnarray*}

Therefore $\displaystyle -\frac{T''(t)}{T(t)}$ must be constant. Suppose that $\displaystyle \frac{T''(t)}{T(t)} = -\lambda$. The existence of such a $\lambda$ is uncertain at this point. So we consider the following eigenvalue problem.

Find a real number $\lambda$ and a $x \in V$ with $x \neq 0$ such that
\begin{eqnarray*}
	b(x,y) = \lambda c(x,y) \ \ \ \ \textrm{ for each } \ y \in V.
\end{eqnarray*}

A solution to the eigenvalue problem consists of an eigenvalue $\lambda$ with corresponding eigenvector $x$. 

The article \cite{CVV18} is a convenient reference to show that the eigenvalue problem has a solution if the following assumption, aditional to \textbf{A1}, \textbf{A2}, \textbf{A3} and \textbf{A4},  is satisfied:

\begin{itemize}
	\item[] \textbf{A5} - The embedding of $V$ into $W$ is compact.
\end{itemize}

(This assumption expands on the assumptions \textbf{A1}-\textbf{A4} already made in Section 2.2.)

Using these assumptions, the authors of \cite{CVV18} prove that there exists a complete orthonormal sequence of eigenvectors for the eigenvalue problem with a corresponding sequence of eigenvalues. These eigenvalues are positive and the orthogonality is with respect to the bilinear form $c$. In fact it is also orthogonal with respect to the bilinear form $b$,
\begin{eqnarray*}
	b(x_i, x_j) = \lambda_i c(x_i, x_j) = 0, \ \ \ \ \textrm{ for each } \ i \neq j.
\end{eqnarray*}

Also the sequence of normalized eigenvectors ${x_i}$ forms an orthonormal basis in $W$ and sequence of eigenvalues ${\lambda_i}$ is an infinite sequence with $\lambda_n \rightarrow \infty$ as $n \rightarrow \infty$. 

Following these results from \cite{CVV18}, for any $u \in V$, $\displaystyle u = \sum_{i=1}^{\infty} a_i x_i$. These coefficients $a_i$ are generalized Fourier coefficients of $u$ with respect to the eigenvectors $x_i$. Therefore, for any $u \in V$,
\begin{eqnarray*}
	u = \sum_{i=1}^{\infty} a_i x_i = \sum_{i=1}^{\infty} c(u, x_i)x_i.
\end{eqnarray*}

Now that the eigenvalue problem has many solutions, the following ordinary differentiable equation can be considered,
\begin{eqnarray*}
	T_n'' + \lambda T_n = 0. 
\end{eqnarray*}

Since this differential equation is a simple second order differential equation, $T_n(t)$ has the following possible solutions:
\begin{flalign}
	T_n(t) &=  A_n \cos(\sqrt{\lambda_n} t) + B_n \sin(\sqrt{\lambda_n} t) \quad \textrm{ if } \lambda_n > 0, \label{lambda_1}
\end{flalign}

Then combining the solutions of the eigenvalue problem and the differential equation, the formal series solution for the boundary value problem is
\begin{eqnarray}
	u(t) = \sum_{n=1}^{\infty} T_n(t)x_n. \label{eq:1D_Model:ModalAnalysisSeriesSolution}
\end{eqnarray}


\subsection{Validity of series solution}
Consider the following question: When is the formal series solution valid for the vibration problem with initial values $u(0) = u_0$ and $u'(0) = u_1$? To answer this question, we again refer to the article \cite{CVV18}.

In the article, the validity of the series solution is proved using energy norms, where the energy $\mathcal{E}$ of the function $u$ given by \label{sym:Energy}
\begin{eqnarray}
	\mathcal{E} (t) = \frac{1}{2} b(u(t), u(t)) + \frac{1}{2} c(u'(t), u'(t)). \label{eq:1D_Model:ModalAnalysisEnergy}
\end{eqnarray}

Then,
\begin{eqnarray*}
	\mathcal{E}'(t) & = & b(u(t), u'(t)) + c(u'(t), u''(t)),\\
					& = & 0 \ \ \ \ \ \ \ \textrm{ following from \eqref{eq:existence:ProblemGVarHom}}. 
\end{eqnarray*}

Therefore $\mathcal{E}(t) = \mathcal{E}(0)$ for all $t>0$. For the case of weak damping, it can be proved that $\mathcal{E}(t) \leq \mathcal{E}(0)$ for all $t>0$ which is the case of \cite{CVV18}.

Denote the partial sum $u^{N}(t) = \sum_{n=1}^{N} T_{n}(t)w_n$. Ideally
\begin{eqnarray*}
	u_0^{N} = \sum_{n=1}^{N} T_n(0) w_n \ \ \textrm{ and } \ \ u_{1}^{N} =\sum_{n=1}^{N} T'_n(0) w_n,
\end{eqnarray*}
but $T_n$ is not uniquely defined by the differential equation.

Substitution shows that these partial sums with initial conditions $u^N(0) = u^N_0$ and $(u^N)'(0) = u^N_1$ are solutions for \eqref{eq:existence:ProblemGVarHom} in Problem GVar.

The authors then define an error function $u^E_N = u - u^N$. Since both $u$ and $u^N$ satisfies \eqref{eq:existence:ProblemGVarHom}, so does this error function with the following the initial conditions, $u^E_N(0) = u_0 - u^N_0$ and $(u^E_N)'(0) = u_1 - u^N_1$.

Let $\mathcal{E}$ denote the energy of the error function $u^E_N$. Since $\mathcal{E}(t) = \mathcal{E}(0)$ for all $t>0$ it follows that, 
\begin{eqnarray}
	||u(t) -  u^N(t)||_V^2 + ||u'(t) - (u^N)'(t)||^2_W = ||u_0 - u^N_0||_V^2 + ||u_1 - u^N_1||_W^2. \label{eq:inequality}
\end{eqnarray}

Now, $u_0$ and $u_1$ are given in Problem GVar. Therefore the generalized Fourier coefficients for $u_0$ and $u_1$ must be used to compute $u_0^N$ and $u_1^N$. These Fourier coefficients are $\displaystyle u_0^N = \sum_{n=1}^{N} b(u_0, w_n)w_n$ and $\displaystyle u_1^N = \sum_{n=1}^{N} c(u_1, w_n)w_n$.

Then $||u_0 - u^N_0||_V^2 \rightarrow 0$ and $||u_1 - u^N_1||_W^2 \rightarrow 0$ as $N \rightarrow \infty$. Therefore $\mathcal{E}(t) \rightarrow 0$ as $N \rightarrow \infty$ by \eqref{eq:inequality}.


It follows that the partial sums of the series solution converges to the solution $u$ as the initial conditions $u_0^N$ and $u_1^N$ converges to $u_0$ and $u_1$ respectively.

\subsection{Comparison of models}

In this dissertation, the objective is to compare different linear beam and plate models for use in applications. The first of the comparisons are in Section \ref{sec:SP06}. This section is a discussion of the article \cite{SP06} comparing a Timoshenko beam model and a three dimensional beam model to empirical results. The authors of this article use the natural frequencies (equivalently the eigenvalues) to compare the different models.

Sections 4.5 and 6.2 are a discussion and extension of the article \cite{LVV09}. This article compares a Timoshenko beam model to a two-dimensional beam model. This is then extended to a comparison of a two-dimensional beam model to a three-dimensional beam model as well as a comparison of a two-dimensional plate model to a three-dimensional plate model.

Being able to express the solutions as valid series solutions, enable us to compare the different models by only considering the eigenvalues and eigenfunctions of the different models. This is explained in detail in \cite{CVV18}.

In \cite{CVV18}, the authors consider a beam model and wave equation model for a vibrating string. Suppose that for the same initial conditions, $u_b$ and $u_w$ are the exact solutions. A comparison of the exact solutions is not possible, but the partial sums can be compared if the bounds for the errors $u_b - u^N_b$ and $u_w - u^N_w$ can be guaranteed.

\end{document}

\textcolor{red}{Voorstel:}\\
\textcolor{red}{*********************}\\
Using these complete product spaces, \eqref{def_of_u} and the variational form \eqref{var_form_timo} defined in Section \ref{ssec:1D_Model:VariationalForm}, the weak variational form for the cantilever Timoshenko beam is defined as Problem T-2W.\\
\textcolor{red}{*********************}

\begin{proof}
	Suppose $f,g \in R[a,b]$. Then
	\begin{eqnarray}
		\left| \int_a^b fg \right| & \leq & ||f||\ ||g|| \ \ \ \ \textrm{ by Cauchy-Swartz Inequality.} \label{CSI}
	\end{eqnarray}

	Suppose that $x^*$ is the zero of $u$ on the interval $[a,b]$. Then since $u \in C^1[0,1]$, by the Fundamental Theorem of Calculus
	\begin{eqnarray*}
		u(x) = \int_{x^*}^x u' \ \ \ \ \textrm{or} \ \ \ \ \ u(x) = -\int^{x^*}_x u'
	\end{eqnarray*}

	So by \eqref{CSI} with $g = 1$, it follows that $|u(x)| \leq ||u'||\sqrt{b-a}$. So it is clear that $||u'||\sqrt{b-a}$ is an upper bound for $|u|$, hence
	\begin{eqnarray}
		||u||_{\sup} \leq ||u'||\sqrt{b-a} \label{sup}
	\end{eqnarray}

	It is easy to show that $||u|| \leq \sqrt{b-a} ||u||_{\sup}$. It then follows from \eqref{sup} that $||u|| \leq (b-a) ||u'||$.
\end{proof}



\subsection{General vibration problem old}

\textcolor{red}{CVV18 word verwys, kyk onder}\\

Consider Problem GVar in Section 2.2.\\

Suppose there exists a number $\lambda$ and $x \in V$, $x \neq 0$ such that
\begin{eqnarray*}
	b(x,y) = \lambda c(x,y) \ \ \ \ \textrm{ for each } \ y \in V.
\end{eqnarray*}

Then, $u(t) = T(t)x$ is a solution of the homogeneous problem
\begin{eqnarray*}
	c(u''(t),v) + b(u(t),v) = 0
\end{eqnarray*}
if and only if
\begin{eqnarray}
	T''  + \lambda T = 0.
\end{eqnarray}

General solution $T(t) = c_1 \sin(\omega t) + c_2 \cos(\omega t)$ with $\omega = \sqrt{\lambda}$.\\

The function $u$ is a natural mode of vibration and $\omega$ natural angular frequency. Natural frequency is $\frac{\omega}{2 \pi}$.

\subsection*{Series solution}\label{ssec:existence:SeriesSolution}
Next we consider the question: When is the formal series solution valid for the vibration problem **(1)** with initial values $u(0) = u_0$ and $u'(0) = u_1$?\\

Consider the partial sum $u^{N}(t) = \sum_{n=1}^{N} T_{n}(t)w_n$. Let
\begin{eqnarray*}
	u_0^{N} = \sum_{n=1}^{N} A^*_n w_n \ \ \textrm{ and } \ \ u_{1}^{N} =\sum_{n=1}^{N} B^*_n w_n
\end{eqnarray*} where $A^*_n = T_n(0)$ and $B^*_n = T'_n(0)$.\\

\textbf{Assumption A5} The embedding of $V$ into $W$ is compact.\\


\textcolor{red}{************************************}\\
\textcolor{red}{VOORSTEL:}

Consider Problem GVar defined in Section 2.2.\\

Suppose there exists a number $\lambda$ and $x \in V$, $x \neq 0$ such that
\begin{eqnarray}
	b(x,y) = \lambda c(x,y) \ \ \ \ \textrm{ for each } \ y \in V. \label{eq:1D_Model:ModalAnalysisEquation}
\end{eqnarray}

nie seperation of variables nie, eers question dan cvv. Kry modal solutions, kan ons dit nou gebruik om algemene solution te kry?
\textcolor{red}{Following the method of seperation of variables,} $u(t) = T(t)x$ is a solution of the homogeneous problem
\begin{eqnarray}
	c(u''(t),v) + b(u(t),v) = 0
\end{eqnarray}
if and only if
\begin{eqnarray}
	T''  + \lambda T = 0. \label{eq:1D_Model:ModalAnalysisT1}
\end{eqnarray}

\textcolor{red}{A general solution for \eqref{eq:1D_Model:ModalAnalysisT} is} $T(t) = c_1 \sin(\omega t) + c_2 \cos(\omega t)$ with $\omega = \sqrt{\lambda}$.\\

The function $u$ is a natural mode of vibration and $\omega$ natural angular frequency. Natural frequency is $\frac{\omega}{2 \pi}$.\\

From \cite{CVV18}, it follows that a formal series solution for \eqref{eq:1D_Model:ModalAnalysisEquation} can be expressed as
\begin{eqnarray}
	u(t) = \sum_{n=1}^{\infty} T_n(t)x_n \label{eq:1D_Model:ModalAnalysisSeriesSolution}
\end{eqnarray}

The following question is considered: When is the formal series solution valid for the vibration problem with initial values $u(0) = u_0$ and $u'(0) = u_1$? To answer this question, consider \cite{CVV18}.\\

\subsubsection{Validity of series solution}
Consider the partial sum $u^{N}(t) = \sum_{n=1}^{N} T_{n}(t)w_n$. Let
\begin{eqnarray*}
	u_0^{N} = \sum_{n=1}^{N} A^*_n w_n \ \ \textrm{ and } \ \ u_{1}^{N} =\sum_{n=1}^{N} B^*_n w_n
\end{eqnarray*} where $A^*_n = T_n(0)$ and $B^*_n = T'_n(0)$.\\

In \cite{CVV18}, the validity of the series solution is proved using the energy norms. The energy $\mathcal{E}$ a function $u$ given by
\begin{eqnarray}
	\mathcal{E} (t) = \frac{1}{2} b(u(t), u(t)) + \frac{1}{2} c(u'(t), u'(t)). \label{eq:1D_Model:ModalAnalysisEnergy}
\end{eqnarray}

The partial sums with initial conditions $u^N(0) = u^N_0$ and $(u^N)'(0) = u^N_1$ are solutions to Problem GVar.\\

In \cite{CVV18}, a error function is defined $u^E_N = u - u^N$. Following the initial conditions, $u^E_N = u_0 - u^N_0$ and $(u^E_N)'(0) = u_1 - u^N_1$.\\

Let $\mathcal{E}$ denote the energy of the error function $u^E_N$. In \cite{CVV18} it is proven that $\mathcal{E}(t) \leq \mathcal{E}(0)$. Using this inequality it follows that, \[||u(t) -  u^N(t)||_V^2 + ||u'(t) - (u^N)'(t)||^2_W \leq ||u_0 - u^N_0||_V^2 + ||u_1 - u^N_1||_W^2. \]

The authors of \cite{CVV18} prove that $||u_0 - u^N_0||_V^2 \rightarrow 0$ and $||u_1 - u^N_1||_W^2 \rightarrow 0$ as $N \rightarrow \infty$. Therefore $\mathcal{E}(t) \rightarrow 0$ as $N \rightarrow \infty$.\\

It follows that the partial series solution converges to the solution $u$ as the initial conditions $u_0^N$ and $u_1^N$ converges to $u_0$ and $u_1$ respectively.\\

\textcolor{red}{****************OLD*************}

Hele dissertation gaan oor comparison of models.


In a later chapter of this dissertation, different dimensional models are compared with each other. Since the formal series solution is a linear combination of eigenvalues and eigenfunctions, it is only required to compare the eigenvalues and eigenfunctions of the different models.\\

Using the results of this subsection, it can be shown that for specific initial conditions, the formal series solution is valid for the different models. So if the two models are disturbed in the exact same way, the formal series solution is valid for b


\subsubsection{Comparison of models}
Hele disseratoin gaan oor comparison of models. \\


In a later chapter of this dissertation, different dimensional models are compared with each other. Since the formal series solution is a linear combination of eigenvalues and eigenfunctions, it is only required to compare the eigenvalues and eigenfunctions of the different models.\\

Using the results of this subsection, it can be shown that for specific initial conditions, the formal series solution is valid for the different models. So if the two models are disturbed in the exact same way, the formal series solution is valid for both models, and hence the eigenvalues and eigenfunctions can be used to compare the models.\\


\textcolor{red}{**************OLD VERSION******************}

\sout{It is well known} that the eigenvalue problem like Problem T-1E has infinitely many solutions $\langle \tilde{w}_n, \tilde{\phi}_n \rangle$ and corresponding sequence of eigenvalues $\lambda_n$. For the pinned-pinned beam (Problem T-1), $\lambda_n > 0$ for all $n > 0$ (this is also true for Problem T-2 and Problem T-4). A convenient reference is \cite{VV06}.

The eigenvalues for the pinned-pinned beam model is the first example in \cite{VV06}. In this dissertation, the pinned-pinned model problem is referred to as Problem T-1 (eigenvalue problem Problem T-1E). Obviously, since $\langle 0 \ 1\rangle$ satisfies the boundary conditions, it is an eigenfunction with corresponding eigenvalue $\alpha$. In \cite{VV06} the authors proved the following theorem.

\newtheorem{ThmVV06_2}{Theorem}
\begin{ThmVV06_2}
	If $\langle u \ \phi\rangle$ is a nonconstant eigenfunction of Problem T-1E, then $\langle u \ \phi\rangle = \langle\sin k \pi x \ A_k \cos k \pi x\rangle$ which satisfies the boundary conditions. $A_k$ is a constant depending on the integer $k$ and the eigenvalue $\lambda_k$. Also all eigenvalues less than $\alpha$ are simple eigenvalues.
\end{ThmVV06_2}

\textcolor{red}{********************************} 